\documentclass[a4paper,12pt,openright]{book}
%\documentclass[a4paper, 12pt, openright, draft]{book}  Nalogo preverite tudi z opcijo draft, ki pokaže, katere vrstice so predolge! Pozor, v draft opciji, se slike ne pokažejo!
 
\usepackage[utf8]{inputenc}
\usepackage[slovene,english]{babel}
\usepackage[pdftex]{graphicx}
\usepackage{fancyhdr}
\usepackage{amssymb}
\usepackage{amsmath}           % eqref, npr.
\usepackage{hyperxmp}
\usepackage[hyphens]{url}
\usepackage{csquotes}
\usepackage[pdftex, colorlinks=true,
						citecolor=black, filecolor=black, 
						linkcolor=black, urlcolor=black,
						pdfproducer={LaTeX}, pdfcreator={LaTeX}]{hyperref}

\usepackage{color}
\usepackage{soul}
\usepackage{array}

\usepackage[
backend=biber,
style=numeric,
sorting=nty,
]{biblatex}

\addbibresource{bibliography.bib}


%%%%%%%%%%%%%%%%%%%%%%%%%%%%%%%%%%%%%%%%
%	DIPLOMA INFO
%%%%%%%%%%%%%%%%%%%%%%%%%%%%%%%%%%%%%%%%
\newcommand{\ttitle}{Primerjava sistemskih klicev Linux in Windows}
\newcommand{\ttitleEn}{The comparison of Linux and Windows system calls}
\newcommand{\tsubject}{\ttitle}
\newcommand{\tsubjectEn}{\ttitleEn}
\newcommand{\tauthor}{Miha Meglič}
\newcommand{\tkeywords}{sistemski klic, linux, windows, operacijski sistem}
\newcommand{\tkeywordsEn}{system call, linux, windows, operating system}

%%%%%%%%%%%%%%%%%%%%%%%%%%%%%%%%%%%%%%%%
%	HYPERREF SETUP
%%%%%%%%%%%%%%%%%%%%%%%%%%%%%%%%%%%%%%%%
\hypersetup{pdftitle={\ttitle}}
\hypersetup{pdfsubject=\ttitleEn}
\hypersetup{pdfauthor={\tauthor}}
\hypersetup{pdfkeywords=\tkeywordsEn}

%%%%%%%%%%%%%%%%%%%%%%%%%%%%%%%%%%%%%%%%
% postavitev strani
%%%%%%%%%%%%%%%%%%%%%%%%%%%%%%%%%%%%%%%%  

\addtolength{\marginparwidth}{-20pt} % robovi za tisk
\addtolength{\oddsidemargin}{40pt}
\addtolength{\evensidemargin}{-40pt}

\renewcommand{\baselinestretch}{1.3} % ustrezen razmik med vrsticami
\setlength{\headheight}{15pt}        % potreben prostor na vrhu
\renewcommand{\chaptermark}[1]%
{\markboth{\MakeUppercase{\thechapter.\ #1}}{}} \renewcommand{\sectionmark}[1]%
{\markright{\MakeUppercase{\thesection.\ #1}}} \renewcommand{\headrulewidth}{0.5pt} \renewcommand{\footrulewidth}{0pt}
\fancyhf{}
\fancyhead[LE,RO]{\sl \thepage} 
%\fancyhead[LO]{\sl \rightmark} \fancyhead[RE]{\sl \leftmark}
\fancyhead[RE]{\sc \tauthor}
\fancyhead[LO]{\sc Diplomska naloga}


\newcommand{\BibLaTeX}{{\sc Bib}\LaTeX}
\newcommand{\BibTeX}{{\sc Bib}\TeX}

%%%%%%%%%%%%%%%%%%%%%%%%%%%%%%%%%%%%%%%%
% naslovi
%%%%%%%%%%%%%%%%%%%%%%%%%%%%%%%%%%%%%%%%  

\newcommand{\autfont}{\Large}
\newcommand{\titfont}{\LARGE\bf}
\newcommand{\clearemptydoublepage}{\newpage{\pagestyle{empty}\cleardoublepage}}
\setcounter{tocdepth}{1}

%%%%%%%%%%%%%%%%%%%%%%%%%%%%%%%%%%%%%%%%
% konstrukti
%%%%%%%%%%%%%%%%%%%%%%%%%%%%%%%%%%%%%%%%  
\newtheorem{izrek}{Izrek}[chapter]
\newtheorem{trditev}{Trditev}[izrek]
\newenvironment{dokaz}{\emph{Dokaz.}\ }{\hspace{\fill}{$\Box$}}


%%%%%%%%%%%%%%%%%%%%%%%%%%%%%%%%%%%%%%%%%%%%%%%%%%%%%%%%%%%%%%%%%%%%%%%%%%%%%%%
%% PDF-A
%%%%%%%%%%%%%%%%%%%%%%%%%%%%%%%%%%%%%%%%%%%%%%%%%%%%%%%%%%%%%%%%%%%%%%%%%%%%%%%

%%%%%%%%%%%%%%%%%%%%%%%%%%%%%%%%%%%%%%%% 
% define medatata
%%%%%%%%%%%%%%%%%%%%%%%%%%%%%%%%%%%%%%%% 
\def\Title{\ttitle}
\def\Author{\tauthor, miha@meglic.dev}
\def\Subject{\ttitleEn}
\def\Keywords{\tkeywordsEn}

%%%%%%%%%%%%%%%%%%%%%%%%%%%%%%%%%%%%%%%% 
% \convertDate converts D:20080419103507+02'00' to 2008-04-19T10:35:07+02:00
%%%%%%%%%%%%%%%%%%%%%%%%%%%%%%%%%%%%%%%% 
\def\convertDate{%
    \getYear
}

{\catcode`\D=12
 \gdef\getYear D:#1#2#3#4{\edef\xYear{#1#2#3#4}\getMonth}
}
\def\getMonth#1#2{\edef\xMonth{#1#2}\getDay}
\def\getDay#1#2{\edef\xDay{#1#2}\getHour}
\def\getHour#1#2{\edef\xHour{#1#2}\getMin}
\def\getMin#1#2{\edef\xMin{#1#2}\getSec}
\def\getSec#1#2{\edef\xSec{#1#2}\getTZh}
\def\getTZh +#1#2{\edef\xTZh{#1#2}\getTZm}
\def\getTZm '#1#2'{%
    \edef\xTZm{#1#2}%
    \edef\convDate{\xYear-\xMonth-\xDay T\xHour:\xMin:\xSec+\xTZh:\xTZm}%
}

%\expandafter\convertDate\pdfcreationdate 

%%%%%%%%%%%%%%%%%%%%%%%%%%%%%%%%%%%%%%%%
% get pdftex version string
%%%%%%%%%%%%%%%%%%%%%%%%%%%%%%%%%%%%%%%% 
\newcount\countA
\countA=\pdftexversion
\advance \countA by -100
\def\pdftexVersionStr{pdfTeX-1.\the\countA.\pdftexrevision}


%%%%%%%%%%%%%%%%%%%%%%%%%%%%%%%%%%%%%%%%
% XMP data
%%%%%%%%%%%%%%%%%%%%%%%%%%%%%%%%%%%%%%%%  
\usepackage{xmpincl}
%\includexmp{pdfa-1b}

%%%%%%%%%%%%%%%%%%%%%%%%%%%%%%%%%%%%%%%%
% pdfInfo
%%%%%%%%%%%%%%%%%%%%%%%%%%%%%%%%%%%%%%%%  
\pdfinfo{%
    /Title    (\ttitle)
    /Author   (\tauthor, miha@meglic.dev)
    /Subject  (\ttitleEn)
    /Keywords (\tkeywordsEn)
    /ModDate  (\pdfcreationdate)
    /Trapped  /False
}

%%%%%%%%%%%%%%%%%%%%%%%%%%%%%%%%%%%%%%%%
% znaki za copyright stran
%%%%%%%%%%%%%%%%%%%%%%%%%%%%%%%%%%%%%%%%  

\newcommand{\CcImageCc}[1]{%
	\includegraphics[scale=#1]{resources/cc_cc_30.pdf}%
}
\newcommand{\CcImageBy}[1]{%
	\includegraphics[scale=#1]{resources/cc_by_30.pdf}%
}
\newcommand{\CcImageSa}[1]{%
	\includegraphics[scale=#1]{resources/cc_sa_30.pdf}%
}

%%%%%%%%%%%%%%%%%%%%%%%%%%%%%%%%%%%%%%%%%%%%%%%%%%%%%%%%%%%%%%%%%%%%%%%%%%%%%%%
%%%%%%%%%%%%%%%%%%%%%%%%%%%%%%%%%%%%%%%%%%%%%%%%%%%%%%%%%%%%%%%%%%%%%%%%%%%%%%%

\begin{document}
\selectlanguage{slovene}
\frontmatter
\setcounter{page}{1} %
\renewcommand{\thepage}{}       % preprečimo težave s številkami strani v kazalu

%%%%%%%%%%%%%%%%%%%%%%%%%%%%%%%%%%%%%%%%
%naslovnica
\thispagestyle{empty}%
\begin{center}
	{\large\sc Univerza v Ljubljani\\
		Fakulteta za računalništvo in informatiko\\
	}
	\vskip 10em
	{\autfont \tauthor\par}
	{\titfont \ttitle \par}
	{\vskip 3em \textsc{DIPLOMSKO DELO\\[5mm]
		UNIVERZITETNI  ŠTUDIJSKI PROGRAM\\ PRVE STOPNJE\\ RAČUNALNIŠTVO IN INFORMATIKA}\par}
	\vfill\null
	{\large \textsc{Mentor}: doc. dr. Jurij Mihelič\par}
	{\vskip 2em \large Ljubljana, \the\year \par}
\end{center}
% prazna stran
%\clearemptydoublepage      
% izjava o licencah itd. se izpiše na hrbtni strani naslovnice

%%%%%%%%%%%%%%%%%%%%%%%%%%%%%%%%%%%%%%%%
%copyright stran
%%%%%%%%%%%%%%%%%%%%%%%%%%%%%%%%%%%%%%%%
\newpage
\thispagestyle{empty}

\vspace*{5cm}
{\small \noindent
	To delo je ponujeno pod licenco \textit{Creative Commons Priznanje avtorstva-Deljenje pod enakimi pogoji 2.5 Slovenija} (ali novej\v so razli\v cico).
	To pomeni, da se tako besedilo, slike, grafi in druge sestavine dela kot tudi rezultati diplomskega dela lahko prosto distribuirajo,
	reproducirajo, uporabljajo, priobčujejo javnosti in predelujejo, pod pogojem, da se jasno in vidno navede avtorja in naslov tega
	dela in da se v primeru spremembe, preoblikovanja ali uporabe tega dela v svojem delu, lahko distribuira predelava le pod
	licenco, ki je enaka tej.
	Podrobnosti licence so dostopne na spletni strani \href{http://creativecommons.si}{creativecommons.si} ali na Inštitutu za
	intelektualno lastnino, Streliška 1, 1000 Ljubljana.
																																																																																																																																																																																																																																																	
	\vspace*{1cm}
	\begin{center}% 0.66 / 0.89 = 0.741573033707865
		\CcImageCc{0.741573033707865}\hspace*{1ex}\CcImageBy{1}\hspace*{1ex}\CcImageSa{1}%
	\end{center}
}

\vspace*{1cm}
{\small \noindent
	Izvorna koda diplomskega dela, njeni rezultati in v ta namen razvita programska oprema je ponujena pod licenco GNU General Public License,
	različica 3 (ali novejša). To pomeni, da se lahko prosto distribuira in/ali predeluje pod njenimi pogoji.
	Podrobnosti licence so dostopne na spletni strani \url{http://www.gnu.org/licenses/}.
}

\vfill
\begin{center} 
	\ \\ \vfill
	{\em
		Besedilo je oblikovano z urejevalnikom besedil \LaTeX.}
\end{center}

% prazna stran
\clearemptydoublepage

%%%%%%%%%%%%%%%%%%%%%%%%%%%%%%%%%%%%%%%%
% stran 3 med uvodnimi listi
\thispagestyle{empty}
\
\vfill

\bigskip
\noindent\textbf{Kandidat:} \tauthor\\
\noindent\textbf{Naslov:} \ttitle\\
\noindent\textbf{Vrsta naloge:} Diplomska naloga na univerzitetnem programu prve stopnje Računalništvo in informatika \\
\noindent\textbf{Mentor:} doc. dr. Jurij Mihelič

\bigskip
\noindent\textbf{Opis:}\\
Besedilo teme diplomskega dela študent prepiše iz študijskega informacijskega sistema, kamor ga je vnesel mentor. 
V nekaj stavkih bo opisal, kaj pričakuje od kandidatovega diplomskega dela. 
Kaj so cilji, kakšne metode naj uporabi, morda bo zapisal tudi ključno literaturo.

\bigskip
\noindent\textbf{Title:} \ttitleEn

\bigskip
\noindent\textbf{Description:}\\
opis diplome v angleščini

\vfill



\vspace{2cm}

% prazna stran
\clearemptydoublepage

% zahvala
\thispagestyle{empty}\mbox{}\vfill\null\it%
\noindent
Na tem mestu zapišite, komu se zahvaljujete za pomoč pri izdelavi diplomske naloge oziroma pri vašem študiju nasploh. Pazite, da ne boste koga pozabili. Utegnil vam bo zameriti. Temu se da izogniti tako, da celotno zahvalo izpustite.
\rm\normalfont

% prazna stran
\clearemptydoublepage


%%%%%%%%%%%%%%%%%%%%%%%%%%%%%%%%%%%%%%%%
% kazalo
\pagestyle{empty}
\def\thepage{}% preprečimo težave s številkami strani v kazalu
\tableofcontents{}


% prazna stran
\clearemptydoublepage

%%%%%%%%%%%%%%%%%%%%%%%%%%%%%%%%%%%%%%%%
% seznam kratic

\chapter*{Seznam uporabljenih kratic}

\noindent\begin{tabular}{p{0.11\textwidth}|p{.39\textwidth}|p{.39\textwidth}}    % po potrebi razširi prvo kolono tabele na račun drugih dveh!
{\bf kratica}	& {\bf angleško}							& {\bf slovensko} \\ \hline
{\bf OS}		& Operating System							& operacijski sistem \\
{\bf API}		& Application Programming Interface			& vmesnik uporabniškega programa \\
{\bf POSIX}		& Portable Operating System Interface		& prenosni vmesnik za operacijski sistem \\
\end{tabular}


% prazna stran
\clearemptydoublepage

%%%%%%%%%%%%%%%%%%%%%%%%%%%%%%%%%%%%%%%%
% povzetek
\addcontentsline{toc}{chapter}{Povzetek}
\chapter*{Povzetek}

\noindent\textbf{Naslov:} \ttitle
\bigskip

\noindent\textbf{Avtor:} \tauthor
\bigskip

%\noindent\textbf{Povzetek:} 
\noindent V vzorcu je predstavljen postopek priprave diplomskega dela z uporabo okolja \LaTeX. Vaš povzetek mora sicer vsebovati približno 100 besed, ta tukaj je odločno prekratek.
Dober povzetek vključuje: (1) kratek opis obravnavanega problema, (2) kratek opis vašega pristopa za reševanje tega problema in (3) (najbolj uspešen) rezultat ali prispevek diplomske naloge.

\bigskip

\noindent\textbf{Ključne besede:} \tkeywords.
% prazna stran
\clearemptydoublepage

%%%%%%%%%%%%%%%%%%%%%%%%%%%%%%%%%%%%%%%%
% abstract
\selectlanguage{english}
\addcontentsline{toc}{chapter}{Abstract}
\chapter*{Abstract}

\noindent\textbf{Title:} \ttitleEn
\bigskip

\noindent\textbf{Author:} \tauthor
\bigskip

%\noindent\textbf{Abstract:} 
\noindent This sample document presents an approach to typesetting your BSc thesis using \LaTeX. 
A proper abstract should contain around 100 words which makes this one way too short.
\bigskip

\noindent\textbf{Keywords:} \tkeywordsEn.
\selectlanguage{slovene}
% prazna stran
\clearemptydoublepage

%%%%%%%%%%%%%%%%%%%%%%%%%%%%%%%%%%%%%%%%
\mainmatter
\setcounter{page}{1}
\pagestyle{fancy}

\chapter{Uvod}


\chapter{Teorija}

Moderni računalniki vsebujejo mnogo različnih komponent in se povezujejo na razno razne vhodno izhodne naprave.
Pri takem obsegu in kompleksnosti računalniških sistemov je praktično nepredstavljivo, da bi vsak programer poznal vse podrobnosti delovanja vseh komponent.
Zato se je pojavila potreba po programu, katerega naloga je upravljanje z viri sistema in abstrakcija dostopa do le teh -- \textbf{operacijski sistem}.

Večina modernih procesorjev ima vsaj dva načina delovanja: \textbf{uporabniški} in \textbf{jedrni} (oz. privilegiran) način.
Operacijski sistem se (večinoma) izvaja v jedrnem načinu, kjer ima dostop do vseh virov sistema in lahko izvede kateri koli ukaz v ukaznem naboru procesorja.
Preostanek programske opreme pa se izvaja v uporabniškem načinu, kjer ima dostop le do omejenega ukaznega nabora.

Kot smo že omenili, je naloga operacijskega sistema upravljanje z viri sistema in abstrakcija dostopa do strojne opreme.
Upravljanje z viri je tako uporabniku kot tudi programerju skrito in v večini transparentno, saj se izvaja avtomatsko in v ozadju.
Abstrakcija dostopa do strojne opreme pa je implementirana s \textbf{sistemskimi klici}.
\cite{Tanenbaum_Bos_2023}

\section{Sistemski klici}

Sistemski klici so vmesnik do storitev, ki jih ponuja jedro operacijskega sistema, kot na primer ustvarjanje procesov, branje in pisanje datotek, komunikacijo med procesi, itd.
Ker torej sistemski klici prehajajo med uporabniškim in jedrnim načinom procesorja, je njihova implementacija odvisna od arhitekture procesorja in njegovega nabora ukazov.

Ker pa jih uporabniški programi potrebujejo dokaj pogosto, so v večini operacijskih sistemov izpostavljeni preko knjižnice ali API-ja.
Najbolj pogosto uporabljena API-ja za aplikacijske programerje sta \textbf{POSIX}, za sisteme, ki sledijo POSIX standardu, npr. Unix, Linux in macOS, in \textbf{Windows API}, za Windows sisteme.
Programer do API-ja dostopa preko knjižnice, ki jo ponuja operacijski sistem -- npr. libc za programski jezik C, v primeru Linuxa.
Tu je omembe vredno tudi, da imena funkcij v knjižnici niso nujno enaka imenom sistemskih klicev, ki jih uporablja sistem.
\cite{Silberschatz_Galvin_Gagne_2018}

Kljub temu, da iz uporabniške oz. programske strani sistemski klici izgledajo kot navadne funkcije, je njihova implementacija popolnoma drugačna:
\begin{enumerate}
	\item Uporabniški proces (oz. knjižnica, ki jo uporablja) shrani številko sistemskega klica in argumente funkcije v registre procesorja -- registri se razlikujejo glede na arhitekturo in OS.
	\item Uporabniški proces sproži past, ki preklopi procesor v jedrni način in požene prekinitveni servisni program (ISR), ki ga je definiral operacijski sistem.
	\item Jedro izvede zahtevan sistemski klic in vrne rezultat -- rezultat se zapiše v register iz katerega ga uporabniški program lahko prebere.
	\item Procesor preklopi nazaj v uporabniški način in nadaljuje izvajanje uporabniškega procesa. \cite{Tanenbaum_Bos_2023}
\end{enumerate}

Preden pa se podamo v primerjave posameznih sistemskih klicev, si poglejmo še, kakšne so implementacijske razlike med sistemskimi klici v Linuxu in Windowsu.
Tu še enkrat poudarjam, da so razlike med sistemskimi klici odvisne od arhitekture procesorja.
V vseh konkretnih primerih bom uporabljal ukazni nabor in imena registrov arhitekture \textbf{x86}.

\subsection{Linux -- POSIX API}

Ker je Linux jedro odprtokodno, si lahko pogledamo tabelo sistemskih klicev, ki jih ponuja na \textbf{x86-64} arhitekturi.
Ta je dostopna v datoteki \texttt{arch/x86} \texttt{/entry/syscalls/syscall\_64.tbl} v \href{https://github.com/torvalds/linux}{izvorni kodi jedra}.
Tu lahko opazimo, da je arhitektura procesorja del poti do datoteke in če pogledamo v druge imenike v \texttt{arch}, lahko vidimo, da ima vsaka implementirana arhitektura med drugim tudi svojo tabelo sistemskih klicev.

Omenili smo že, da Linux implementira POSIX standard, ki definira procedure, ki naj bi jih implementiral operacijski sistem.
V veliki večini te procedure zahtevajo uporabo sistemskih, vendar je potrebno poudariti, da povezava ni nujno ena na ena.
Lahko se zgodi, da neka procedura sploh ne zahteva uporabe sistemskih klicev, v katerem primeru je implementirana v uporabniškem načinu, ali pa zahteva uporabo več sistemskih klicev.
V redkem primeru se lahko zgodi, da več funkcij uporablja isti, bolj splošen, sistemski klic.

\subsection{Windows -- Windows API}

Windows je, v kontrastu z Linuxom, zaprtokodni operacijski sistem, zato je težje priti do podatkov o njegovi implementaciji.
Ena izmed glavnih razlik je, da Windows omogoča dostop do sistemskih klicev izključno preko Windows API-ja.
Torej, če smo v Linuxu lahko pokukali v tabelo sistemskih klicev in jih lahko celo poklicali direktno z zbirnim jezikom, je v Windowsu to nemogoče.

Še ena velika razlika, ki jo bomo srečavali v naslednjem poglavju, je število API procedur.
Windows API izpostavi ogromno procedur, v rangu več tisoč, medtem ko POSIX API izpostavi le nekaj sto.
To je posledica več faktorjev -- pogosto več funkcij uporablja isti sistemski klic, veliko pa je tudi funkcij, ki so v celoti implementirane v uporabniškem načinu.

Ker Windows ne izpostavi sistemskih klicev direktno, se lahko le-te spreminjajo med posameznimi verzijami operacijskega sistema.
To pomeni, da je težko zagotovo reči ali je neka funkcija implementirana v uporabniškem ali jedrnem načinu.
\cite{Tanenbaum_Bos_2023}

\chapter{Primerjava sistemskih klicev}

Ker je sistemskih klicev ogromno, jih bomo v nadaljevanju razdelili na šest kategorij, in sicer:
\begin{itemize}
	\item \textbf{procesni nadzor} -- upravljanje procesov,
	\item \textbf{upravljanje datotek} -- branje in pisanje datotek,
	\item \textbf{upravljanje naprav} -- upravljanje z vhodno-izhodnimi napravami,
	\item \textbf{vzdrževanje informacij o sistemu} -- pridobivanje in urejanje informacij o sistemu,
	\item \textbf{komunikacija med procesi} -- dvosmerna med procesna komunikacija,
	\item \textbf{zaščita} -- upravljanje pravic nad datotekami. \cite{Silberschatz_Galvin_Gagne_2018}
\end{itemize}

\section{Procesni nadzor}

Procesni nadzor je ena ključnih funkcij večopravilnega operacijskega sistema.
Ta kategorija vključuje sistemske klice za:
\begin{itemize}
	\item ustvarjanje in uničevanje procesov,
	\item nalaganje in izvajanje programov,
	\item pridobivanje in nastavljanje atributov procesov,
	\item sinhronizacijo procesov,
	\item dodeljevanje in sproščanje pomnilnika. \cite{Silberschatz_Galvin_Gagne_2018}
\end{itemize}

\begin{center}
	\begin{tabular}{ p{3.7cm}|p{2.5cm}|p{6cm} }
		Function & Linux   & Windows \\
		\hline
		Ustvari  & \verb|| & \verb|| \\
		Ustavi   & \verb|| & \verb|| \\
	\end{tabular}
\end{center}

\section{Upravljanje datotek}

Ta kategorija vključuje sistemske klice za:
\begin{itemize}
	\item ustvarjanje in brisanje datotek,
	\item odpiranje in zapiranje datotek,
	\item branje, pisanje in premikanje datotek,
	\item pridobivanje in nastavljanje atributov datotek. \cite{Silberschatz_Galvin_Gagne_2018}
\end{itemize}

\begin{center}
	\begin{tabular}{ p{3.7cm}|p{2.5cm}|p{6cm} }
		Funkcija           & Linux                                                                                          & Windows                                                           \\
		\hline
		Ustvari            & \verb|creat|\newline\verb|mkdir|                                                               & \verb|CreateFile|\newline\verb|CreateDirectory|                   \\
		Premakni *         & \verb|rename|                                                                                  & \verb|MoveFile|                                                   \\
		Odstrani *         & \verb|unlink|\newline\verb|rmdir|                                                              & \verb|DeleteFile|\newline\verb|RemoveDirectory|                   \\
		Odpri              & \verb|open|                                                                                    & \verb|CreateFile|                                                 \\
		Zapri              & \verb|close|                                                                                   & \verb|CloseHandle|                                                \\
		Beri               & \verb|read|                                                                                    & \verb|ReadFile|                                                   \\
		Piši              & \verb|write|                                                                                   & \verb|WriteFile|                                                  \\
		Premakni kazalnik  & \verb|lseek|                                                                                   & \verb|SetFilePointer|                                             \\
		Pot bližnjice     & \verb|readlink|                                                                                & \verb|GetFinalPathNameByHandle|                                   \\
		Ustvari bližnjico & \verb|link|\newline\verb|symlink|                                                              & \verb|CreateHardLink|\newline\verb|CreateSymbolicLink|            \\
		Zakleni            & \verb|flock|                                                                                   & \verb|LockFile|                                                   \\
		Odkleni            & \verb|flock|                                                                                   & \verb|UnlockFile|                                                 \\
		Pridobi atribute   & \verb|fstat|\newline\verb|stat|\newline\verb|lstat|\newline\verb|utime|\newline\verb|getxattr| & \verb|GetFileInformationByHandle|\newline\verb|GetFileAttributes| \\
		Nastavi atribute   & \verb|setxattr|                                                                                & \verb|SetFileInformationByHandle|\newline\verb|SetFileAttributes| \\
	\end{tabular}
\end{center}

t.i. opisnik datoteke (\textit{angl. file descriptor -- abbr. fd})

\section{Upravljanje naprav}

Ta kategorija vključuje sistemske klice za:
\begin{itemize}
	\item zahtevanje in sprostitev naprav,
	\item branje, pisanje in premikanje naprav,
	\item pridobivanje in nastavljanje atributov naprav,
	\item logično povezovanje naprav. \cite{Silberschatz_Galvin_Gagne_2018}
\end{itemize}

Tu pričnemo opažati drastične razlike med arhitekturo jeder Linux in Windows. Osredotočimo se na upravljanje pomnilnika, saj je to ena izmed naprav, ki je skupna obema jedroma, medtem ko so druge odvisne od strojne opreme.

Linux za upravljanje z napravami namreč uporablja datotečni sistem, kar pomeni, da lahko naprave odpiramo in z njimi komuniciramo kot z običajnimi datotekami.
Za večino naprednejših operacij pa nam ponuja sistemski klic \verb|mmap|, ki nam omogoča preslikavo datoteke v pomnilnik in s tem direkten dostop do njenih vsebin.
Na primer za ustvarjanje navideznega pomnilnika (v Windows funkcija \verb|VirtualAlloc|), odpremo datoteko \verb|/dev/zero| in jo preslikamo v pomnilnik z \verb|mmap|.
Za dodeljevanje pomnilnika iz kopice (\textit{angl. heap}), pa lahko uporabimo sistemski klic \verb|brk| ali funkcijo \verb|sbrk|, ki preprosto premakne konec ``podatkovnega prostora'' procesa, ki ga potem lahko uporabimo za poljubne podatke, vendar nam dokumentacija priporoča uporabo funkcije standardne C knjižnice -- \verb|malloc|, ki pa je precej bolj kompleksna in odvisna od implementacije knjižnice.
Pri direktnem nadzoru naprave oz. ``posebne datoteke'' pa nam Linux ponuja funkcije \verb|ioctl| in \verb|fcntl|, ki nam omogočata pošiljanje ukazov napravi.

Windows pa za upravljanje z napravami uporablja mnoge posebne funkcije, ki so del Windows API-ja.
Za navidezni pomnilnik na primer ponuja funkcijo \verb|VirtualAlloc|, za dodeljevanja pomnilnika iz kopice pa funkcije \verb|GlobalAlloc|, \verb|LocalAlloc| in \verb|HeapAlloc| / \verb|HeapCreate|, pri čemer je \verb|HeapAlloc| najbolj osnovna, druge pa nam omogočajo napredne funkcionalnosti kot premikanje dodeljenega pomnilnika in podobno.
Podobno kot Linux, pa nam tudi Windows ponuja funkcijo za preslikavo datoteke v pomnilnik -- \verb|CreateFileMapping| in \verb|MapViewOfFile|.

Predvsem iz primera z dodeljevanjem pomnilnika iz kopice lahko vidimo, da Windows API ponuja več funkcij za isto stvar, kar je posledica večje kompleksnosti in abstrakcije, ki jo ponuja Windows.
To je lahko v nekaterih primerih prednost, saj več abstrakcije (običajno) pomeni boljšo uporabniško izkušnjo (npr. ne potrebujemo sami slediti naslovom v pomnilniku), spet v drugih pa slabost, saj skrije podrobnosti implementacije in nam prepušča manj kontrole.

\section{Vzdrževanje informacij o sistemu}

Ta kategorija vključuje sistemske klice za:
\begin{itemize}
	\item pridobivanje in nastavljanje časa in datuma,
	\item pridobivanje in nastavljanje sistemskih parametrov,
	\item pridobivanje in nastavljanje atributov procesov, datotek in naprav. \cite{Silberschatz_Galvin_Gagne_2018}
\end{itemize}

\begin{center}
	\begin{tabular}{ p{3.7cm}|p{2.5cm}|p{6cm} }
		Function      & Linux   & Windows \\
		\hline
		Pridobi Datum & \verb|| & \verb|| \\
	\end{tabular}
\end{center}

\section{Komunikacija med procesi}

Ta kategorija vključuje sistemske klice za:
\begin{itemize}
	\item ustvarjanje in uničevanje komunikacijskih poti,
	\item pošiljanje in prejemanje sporočil,
	\item prenos statusov,
	\item povezovanje oddaljenih naprav. \cite{Silberschatz_Galvin_Gagne_2018}
\end{itemize}

\begin{center}
	\begin{tabular}{ p{3.7cm}|p{2.5cm}|p{6cm} }
		Function          & Linux   & Windows \\
		\hline
		Ustvari vtičnico & \verb|| & \verb|| \\
	\end{tabular}
\end{center}

\section{Zaščita}

Ta kategorija vključuje sistemske klice za pridobivanje in nastavljanje pravic nad datotekami. \cite{Silberschatz_Galvin_Gagne_2018}

\begin{center}
	\begin{tabular}{ p{3.7cm}|p{2.5cm}|p{6cm} }
		Function       & Linux   & Windows \\
		\hline
		Create Process & \verb|| & \verb|| \\
	\end{tabular}
\end{center}

\chapter{Praktična primerjava}

\section{Metodologija}

Primerjavo izvajamo na identičnih virtualnih sistemih, na Proxmox gostitelju (QEMU), s sledečo konfiguracijo:
\begin{itemize}
	\item i440fx sistemski emulator (UEFI BIOS)
	\item 4 jedra procesorja (Intel\textsuperscript{\textregistered} Core\textsuperscript{\texttrademark} i7-12700H)
	\item 8 GB pomnilnika (Corsair\textsuperscript{\textregistered} Vengeance\textsuperscript{\textregistered} LPX 64GB DDR4 3200 MHz C16)
	\item 100 GB diska (Crucial\textsuperscript{\textregistered} P5 Plus Gen4 NVMe SSD)
\end{itemize}

Za operacijska sistema uporabljamo:
\begin{itemize}
	\item Windows 10 Professional, verzija 22H2
	\item Ubuntu Desktop 22.04 LTS
\end{itemize}
Operacijska sistema sta bila izbrana na podlagi popularnosti in razširjenosti v času pisanja, oba imata namreč največji tržni delež v svoji kategoriji.

\section{Meritve}

\chapter{Zaključek}

%\cleardoublepage
%\addcontentsline{toc}{chapter}{Literatura}

\printbibliography[heading=bibintoc,title={Literatura}]


\end{document}
