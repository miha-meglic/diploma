\documentclass[a4paper,12pt,openright]{book}
%\documentclass[a4paper, 12pt, openright, draft]{book}  Nalogo preverite tudi z opcijo draft, ki pokaže, katere vrstice so predolge! Pozor, v draft opciji, se slike ne pokažejo!
 
\usepackage[utf8]{inputenc}
\usepackage[slovene,english]{babel}
\usepackage[pdftex]{graphicx}
\usepackage{fancyhdr}
\usepackage{amssymb}
\usepackage{amsmath}           % eqref, npr.
\usepackage{hyperxmp}
\usepackage[hyphens]{url}
\usepackage{csquotes}
\usepackage[pdftex, colorlinks=true,
citecolor=black, filecolor=black, 
linkcolor=black, urlcolor=magenta,
pdfproducer={LaTeX}, pdfcreator={LaTeX}]{hyperref}

\usepackage{color}
\usepackage{soul}
\usepackage{array}
\usepackage{tikz}
\usetikzlibrary{positioning,shapes.geometric,arrows.meta,bending,positioning}
\usepackage{listings}

\tikzstyle{arrow} = [thick,->,>=stealth]

\definecolor{codegreen}{rgb}{0,0.6,0}
\definecolor{codegray}{rgb}{0.5,0.5,0.5}
\definecolor{codepurple}{rgb}{0.58,0,0.82}
\definecolor{backcolour}{rgb}{0.95,0.95,0.92}
\definecolor{backcolour_func}{rgb}{0.90,0.90,0.90}

\lstset{
	language=C++,
	backgroundcolor=\color{backcolour},   
    commentstyle=\color{codegreen},
    keywordstyle=\color{magenta},
    numberstyle=\tiny\color{codegray},
    stringstyle=\color{codepurple},
    basicstyle=\ttfamily\footnotesize,
	breaklines=true,
	tabsize=4,
	numbers=left
}

\lstdefinestyle{func}{
	backgroundcolor=\color{white},
	numbers=none,
	frame=single
}

\usepackage[
backend=biber,
style=numeric,
sorting=nty,
]{biblatex}

\addbibresource{bibliography.bib}


%%%%%%%%%%%%%%%%%%%%%%%%%%%%%%%%%%%%%%%%
%	DIPLOMA INFO
%%%%%%%%%%%%%%%%%%%%%%%%%%%%%%%%%%%%%%%%
\newcommand{\ttitle}{Primerjava sistemskih klicev Linux in Windows}
\newcommand{\ttitleEn}{The comparison of Linux and Windows system calls}
\newcommand{\tsubject}{\ttitle}
\newcommand{\tsubjectEn}{\ttitleEn}
\newcommand{\tauthor}{Miha Meglič}
\newcommand{\tkeywords}{sistemski klic, linux, windows, operacijski sistem}
\newcommand{\tkeywordsEn}{system call, linux, windows, operating system}

%%%%%%%%%%%%%%%%%%%%%%%%%%%%%%%%%%%%%%%%
%	HYPERREF SETUP
%%%%%%%%%%%%%%%%%%%%%%%%%%%%%%%%%%%%%%%%
\hypersetup{pdftitle={\ttitle}}
\hypersetup{pdfsubject=\ttitleEn}
\hypersetup{pdfauthor={\tauthor}}
\hypersetup{pdfkeywords=\tkeywordsEn}

%%%%%%%%%%%%%%%%%%%%%%%%%%%%%%%%%%%%%%%%
% postavitev strani
%%%%%%%%%%%%%%%%%%%%%%%%%%%%%%%%%%%%%%%%  

\addtolength{\marginparwidth}{-20pt} % robovi za tisk
\addtolength{\oddsidemargin}{40pt}
\addtolength{\evensidemargin}{-40pt}

\renewcommand{\baselinestretch}{1.3} % ustrezen razmik med vrsticami
\setlength{\headheight}{15pt}        % potreben prostor na vrhu
\renewcommand{\chaptermark}[1]%
{\markboth{\MakeUppercase{\thechapter.\ #1}}{}} \renewcommand{\sectionmark}[1]%
{\markright{\MakeUppercase{\thesection.\ #1}}} \renewcommand{\headrulewidth}{0.5pt} \renewcommand{\footrulewidth}{0pt}
\fancyhf{}
\fancyhead[LE,RO]{\sl \thepage} 
%\fancyhead[LO]{\sl \rightmark} \fancyhead[RE]{\sl \leftmark}
\fancyhead[RE]{\sc \tauthor}
\fancyhead[LO]{\sc Diplomska naloga}


\newcommand{\BibLaTeX}{{\sc Bib}\LaTeX}
\newcommand{\BibTeX}{{\sc Bib}\TeX}

%%%%%%%%%%%%%%%%%%%%%%%%%%%%%%%%%%%%%%%%
% naslovi
%%%%%%%%%%%%%%%%%%%%%%%%%%%%%%%%%%%%%%%%  

\newcommand{\autfont}{\Large}
\newcommand{\titfont}{\LARGE\bf}
\newcommand{\clearemptydoublepage}{\newpage{\pagestyle{empty}\cleardoublepage}}
\setcounter{tocdepth}{1}

%%%%%%%%%%%%%%%%%%%%%%%%%%%%%%%%%%%%%%%%
% konstrukti
%%%%%%%%%%%%%%%%%%%%%%%%%%%%%%%%%%%%%%%%  
\newtheorem{izrek}{Izrek}[chapter]
\newtheorem{trditev}{Trditev}[izrek]
\newenvironment{dokaz}{\emph{Dokaz.}\ }{\hspace{\fill}{$\Box$}}


%%%%%%%%%%%%%%%%%%%%%%%%%%%%%%%%%%%%%%%%%%%%%%%%%%%%%%%%%%%%%%%%%%%%%%%%%%%%%%%
%% PDF-A
%%%%%%%%%%%%%%%%%%%%%%%%%%%%%%%%%%%%%%%%%%%%%%%%%%%%%%%%%%%%%%%%%%%%%%%%%%%%%%%

%%%%%%%%%%%%%%%%%%%%%%%%%%%%%%%%%%%%%%%% 
% define medatata
%%%%%%%%%%%%%%%%%%%%%%%%%%%%%%%%%%%%%%%% 
\def\Title{\ttitle}
\def\Author{\tauthor, miha@meglic.dev}
\def\Subject{\ttitleEn}
\def\Keywords{\tkeywordsEn}

%%%%%%%%%%%%%%%%%%%%%%%%%%%%%%%%%%%%%%%% 
% \convertDate converts D:20080419103507+02'00' to 2008-04-19T10:35:07+02:00
%%%%%%%%%%%%%%%%%%%%%%%%%%%%%%%%%%%%%%%% 
\def\convertDate{%
    \getYear
}

{\catcode`\D=12
 \gdef\getYear D:#1#2#3#4{\edef\xYear{#1#2#3#4}\getMonth}
}
\def\getMonth#1#2{\edef\xMonth{#1#2}\getDay}
\def\getDay#1#2{\edef\xDay{#1#2}\getHour}
\def\getHour#1#2{\edef\xHour{#1#2}\getMin}
\def\getMin#1#2{\edef\xMin{#1#2}\getSec}
\def\getSec#1#2{\edef\xSec{#1#2}\getTZh}
\def\getTZh +#1#2{\edef\xTZh{#1#2}\getTZm}
\def\getTZm '#1#2'{%
    \edef\xTZm{#1#2}%
    \edef\convDate{\xYear-\xMonth-\xDay T\xHour:\xMin:\xSec+\xTZh:\xTZm}%
}

%\expandafter\convertDate\pdfcreationdate 

%%%%%%%%%%%%%%%%%%%%%%%%%%%%%%%%%%%%%%%%
% get pdftex version string
%%%%%%%%%%%%%%%%%%%%%%%%%%%%%%%%%%%%%%%% 
\newcount\countA
\countA=\pdftexversion
\advance \countA by -100
\def\pdftexVersionStr{pdfTeX-1.\the\countA.\pdftexrevision}


%%%%%%%%%%%%%%%%%%%%%%%%%%%%%%%%%%%%%%%%
% XMP data
%%%%%%%%%%%%%%%%%%%%%%%%%%%%%%%%%%%%%%%%  
\usepackage{xmpincl}
%\includexmp{pdfa-1b}

%%%%%%%%%%%%%%%%%%%%%%%%%%%%%%%%%%%%%%%%
% pdfInfo
%%%%%%%%%%%%%%%%%%%%%%%%%%%%%%%%%%%%%%%%  
\pdfinfo{%
    /Title    (\ttitle)
    /Author   (\tauthor, miha@meglic.dev)
    /Subject  (\ttitleEn)
    /Keywords (\tkeywordsEn)
    /ModDate  (\pdfcreationdate)
    /Trapped  /False
}

%%%%%%%%%%%%%%%%%%%%%%%%%%%%%%%%%%%%%%%%
% znaki za copyright stran
%%%%%%%%%%%%%%%%%%%%%%%%%%%%%%%%%%%%%%%%  

\newcommand{\CcImageCc}[1]{%
	\includegraphics[scale=#1]{resources/cc_cc_30.pdf}%
}
\newcommand{\CcImageBy}[1]{%
	\includegraphics[scale=#1]{resources/cc_by_30.pdf}%
}
\newcommand{\CcImageSa}[1]{%
	\includegraphics[scale=#1]{resources/cc_sa_30.pdf}%
}

%%%%%%%%%%%%%%%%%%%%%%%%%%%%%%%%%%%%%%%%%%%%%%%%%%%%%%%%%%%%%%%%%%%%%%%%%%%%%%%
%%%%%%%%%%%%%%%%%%%%%%%%%%%%%%%%%%%%%%%%%%%%%%%%%%%%%%%%%%%%%%%%%%%%%%%%%%%%%%%

\begin{document}
\selectlanguage{slovene}
\frontmatter
\setcounter{page}{1} %
\renewcommand{\thepage}{}       % preprečimo težave s številkami strani v kazalu

%%%%%%%%%%%%%%%%%%%%%%%%%%%%%%%%%%%%%%%%
%naslovnica
\thispagestyle{empty}%
\begin{center}
	{\large\sc Univerza v Ljubljani\\
		Fakulteta za računalništvo in informatiko\\
	}
	\vskip 10em
	{\autfont \tauthor\par}
	{\titfont \ttitle \par}
	{\vskip 3em \textsc{DIPLOMSKO DELO\\[5mm]
		UNIVERZITETNI  ŠTUDIJSKI PROGRAM\\ PRVE STOPNJE\\ RAČUNALNIŠTVO IN INFORMATIKA}\par}
	\vfill\null
	{\large \textsc{Mentor}: doc. dr. Jurij Mihelič\par}
	{\vskip 2em \large Ljubljana, \the\year \par}
\end{center}
% prazna stran
%\clearemptydoublepage      
% izjava o licencah itd. se izpiše na hrbtni strani naslovnice

%%%%%%%%%%%%%%%%%%%%%%%%%%%%%%%%%%%%%%%%
%copyright stran
%%%%%%%%%%%%%%%%%%%%%%%%%%%%%%%%%%%%%%%%
\newpage
\thispagestyle{empty}

\vspace*{5cm}
{\small \noindent
	To delo je ponujeno pod licenco \textit{Creative Commons Priznanje avtorstva-Deljenje pod enakimi pogoji 2.5 Slovenija} (ali novej\v so razli\v cico).
	To pomeni, da se tako besedilo, slike, grafi in druge sestavine dela kot tudi rezultati diplomskega dela lahko prosto distribuirajo,
	reproducirajo, uporabljajo, priobčujejo javnosti in predelujejo, pod pogojem, da se jasno in vidno navede avtorja in naslov tega
	dela in da se v primeru spremembe, preoblikovanja ali uporabe tega dela v svojem delu, lahko distribuira predelava le pod
	licenco, ki je enaka tej.
	Podrobnosti licence so dostopne na spletni strani \href{http://creativecommons.si}{creativecommons.si} ali na Inštitutu za
	intelektualno lastnino, Streliška 1, 1000 Ljubljana.
																																																																																																																																																																																																																																																																																																																																																																																																																																																							
	\vspace*{1cm}
	\begin{center}% 0.66 / 0.89 = 0.741573033707865
		\CcImageCc{0.741573033707865}\hspace*{1ex}\CcImageBy{1}\hspace*{1ex}\CcImageSa{1}%
	\end{center}
}

\vspace*{1cm}
{\small \noindent
	Izvorna koda diplomskega dela, njeni rezultati in v ta namen razvita programska oprema je ponujena pod licenco GNU General Public License,
	različica 3 (ali novejša). To pomeni, da se lahko prosto distribuira in/ali predeluje pod njenimi pogoji.
	Podrobnosti licence so dostopne na spletni strani \url{http://www.gnu.org/licenses/}.
}

\vfill
\begin{center} 
	\ \\ \vfill
	{\em
		Besedilo je oblikovano z urejevalnikom besedil \LaTeX.}
\end{center}

% prazna stran
\clearemptydoublepage

%%%%%%%%%%%%%%%%%%%%%%%%%%%%%%%%%%%%%%%%
% stran 3 med uvodnimi listi
\thispagestyle{empty}
\
\vfill

\bigskip
\noindent\textbf{Kandidat:} \tauthor\\
\noindent\textbf{Naslov:} \ttitle\\
\noindent\textbf{Vrsta naloge:} Diplomska naloga na univerzitetnem programu prve stopnje Računalništvo in informatika \\
\noindent\textbf{Mentor:} doc. dr. Jurij Mihelič

\bigskip
\noindent\textbf{Opis:}\\
Besedilo teme diplomskega dela študent prepiše iz študijskega informacijskega sistema, kamor ga je vnesel mentor. 
V nekaj stavkih bo opisal, kaj pričakuje od kandidatovega diplomskega dela. 
Kaj so cilji, kakšne metode naj uporabi, morda bo zapisal tudi ključno literaturo.

\bigskip
\noindent\textbf{Title:} \ttitleEn

\bigskip
\noindent\textbf{Description:}\\
opis diplome v angleščini

\vfill



\vspace{2cm}

% prazna stran
\clearemptydoublepage

% zahvala
\thispagestyle{empty}\mbox{}\vfill\null\it%
\noindent
Na tem mestu zapišite, komu se zahvaljujete za pomoč pri izdelavi diplomske naloge oziroma pri vašem študiju nasploh. Pazite, da ne boste koga pozabili. Utegnil vam bo zameriti. Temu se da izogniti tako, da celotno zahvalo izpustite.
\rm\normalfont

% prazna stran
\clearemptydoublepage


%%%%%%%%%%%%%%%%%%%%%%%%%%%%%%%%%%%%%%%%
% kazalo
\pagestyle{empty}
\def\thepage{}% preprečimo težave s številkami strani v kazalu
\tableofcontents{}


% prazna stran
\clearemptydoublepage

%%%%%%%%%%%%%%%%%%%%%%%%%%%%%%%%%%%%%%%%
% seznam kratic

\chapter*{Seznam uporabljenih kratic}

\noindent\begin{tabular}{p{0.11\textwidth}|p{.39\textwidth}|p{.39\textwidth}}    % po potrebi razširi prvo kolono tabele na račun drugih dveh!
{\bf kratica} & {\bf angleško}                     & {\bf slovensko}                         \\
\hline
{\bf OS}      & Operating System                    & operacijski sistem                      \\
{\bf API}     & Application Programming Interface   & vmesnik uporabniškega programa          \\
{\bf POSIX}   & Portable Operating System Interface & Prenosni Vmesnik za Operacijske Sisteme \\
{\bf SUS}     & Single UNIX Specification           & Enotna UNIX specifikacija               \\
{\bf ISR}     & Interrupt Service Routine           & prekinitveni servisni program           \\
\end{tabular}


% prazna stran
\clearemptydoublepage

%%%%%%%%%%%%%%%%%%%%%%%%%%%%%%%%%%%%%%%%
% povzetek
\addcontentsline{toc}{chapter}{Povzetek}
\chapter*{Povzetek}

\noindent\textbf{Naslov:} \ttitle
\bigskip

\noindent\textbf{Avtor:} \tauthor
\bigskip

%\noindent\textbf{Povzetek:} 
\noindent V vzorcu je predstavljen postopek priprave diplomskega dela z uporabo okolja \LaTeX. Vaš povzetek mora sicer vsebovati približno 100 besed, ta tukaj je odločno prekratek.
Dober povzetek vključuje: (1) kratek opis obravnavanega problema, (2) kratek opis vašega pristopa za reševanje tega problema in (3) (najbolj uspešen) rezultat ali prispevek diplomske naloge.

\bigskip

\noindent\textbf{Ključne besede:} \tkeywords.
% prazna stran
\clearemptydoublepage

%%%%%%%%%%%%%%%%%%%%%%%%%%%%%%%%%%%%%%%%
% abstract
\selectlanguage{english}
\addcontentsline{toc}{chapter}{Abstract}
\chapter*{Abstract}

\noindent\textbf{Title:} \ttitleEn
\bigskip

\noindent\textbf{Author:} \tauthor
\bigskip

%\noindent\textbf{Abstract:} 
\noindent This sample document presents an approach to typesetting your BSc thesis using \LaTeX. 
A proper abstract should contain around 100 words which makes this one way too short.
\bigskip

\noindent\textbf{Keywords:} \tkeywordsEn.
\selectlanguage{slovene}
% prazna stran
\clearemptydoublepage

%%%%%%%%%%%%%%%%%%%%%%%%%%%%%%%%%%%%%%%%
\mainmatter
\setcounter{page}{1}
\pagestyle{fancy}

\chapter{Uvod}


\chapter{Pregled področja}

Moderni računalniki vsebujejo mnogo različnih komponent -- centralno procesno enoto, pomnilnik in vhodno-izhodne naprave.
Pri takem obsegu in kompleksnosti računalniških sistemov je praktično nepredstavljivo, da bi vsak programer poznal vse podrobnosti delovanja vseh komponent.
Zato se je pojavila potreba po programu, katerega naloga je upravljanje z viri sistema in abstrakcija dostopa do le teh -- \textbf{operacijski sistem}.

\section{Operacijski sistem}

Operacijski sistem iz uporabnikove perspektive stoji med aplikacijami in strojno opremo, kot vidimo na sliki \ref{fig:computer_system_components}.

\begin{figure}[h!]
	\begin{center}
		\begin{tikzpicture}[node distance=1.8cm]
			\tikzstyle{node} = [rectangle, draw=black, fill=blue!15!white, align=center, minimum height=1cm, minimum width=3cm];
			\node (user) [node] {uporabnik};
			\node (application) [node, minimum width=7cm, fill=gray!40!white, below of=user] {aplikacije\\\footnotesize{(brskalniki, urejevalniki besedila, \dots)}};
			\node (os) [node, minimum width=7cm, below of=application] {operacijski sistem};
			\node (hardware) [node, minimum width=7cm, fill=gray!40!white, below of=os] {strojna oprema\\\footnotesize{(CPU, pomnilnik, V/I naprave, \dots)}};
			\draw [thick,<->,>=stealth] (user) -- (application);
			\draw [thick,<->,>=stealth] ([xshift=2cm]application.south) -- ([xshift=2cm]os.north);
			\draw [thick,<->,>=stealth] (application) -- (os);
			\draw [thick,<->,>=stealth] ([xshift=-2cm]application.south) -- ([xshift=-2cm]os.north);
			\draw [thick,<->,>=stealth] ([xshift=2cm]os.south) -- ([xshift=2cm]hardware.north);
			\draw [thick,<->,>=stealth] (os) -- (hardware);
			\draw [thick,<->,>=stealth] ([xshift=-2cm]os.south) -- ([xshift=-2cm]hardware.north);
		\end{tikzpicture}
	\end{center}
	\caption{Abstrakten pogled komponent računalniškega sistema \cite{Silberschatz_Galvin_Gagne_2018}}
	\label{fig:computer_system_components}
\end{figure}

Glavna naloga operacijskega sistema upravljanje z viri sistema in abstrakcija dostopa do strojne opreme \cite{Tanenbaum_Bos_2023}.
Upravljanje z viri je tako uporabniku kot tudi programerju skrito in v večini transparentno, saj se izvaja avtomatsko in v ozadju.
Abstrakcija dostopa do strojne opreme in druge ključne funkcionalnosti pa so implementirane s \textbf{sistemskimi klici}.

\section{Operacijski sistem Linux}

Linux je odprtokodno, UNIXu podobno jedro operacijskega sistema, ki ga je razvil Linux Torvalds.

Seveda pa jedro še ni celoten operacijski sistem, potrebujemo tudi sistemsko uporabniško programsko opremo.
Danes obstaja nešteto mnogo različnih kombinacij jedra in različne programske opreme, ki jih imenujemo distribucije.
Te vsebujejo vse standardne komponente Linux sistema in dodajo administrativna orodja, ki poenostavijo namestitev, nadgradnjo in vzdrževanje sistema ter omogočajo delo z datotečnim sistemom, kreiranje in upravljanje uporabniških računov, administracijo omrežij in mnogo več \cite{Silberschatz_Galvin_Gagne_2018}.
Prav tako ponujajo programe, ki nam omogočajo namestitev dodatne programske opreme in orodij, ki jih uporabnik potrebuje.
Velika večina te programske opreme je bilo razvite v sklopu drugih odprtokodnih projektov kot so GNU, BSD, X Window System, itd.

Eden izmed glavnih ciljev pri razvoju je bila skladnost s standardi POSIX (\textit{angl. Portable Operating System Interface}) in SUS (\textit{angl. Single UNIX Specification}).
To je razlog, da je Linux danes del širokega nabora UNIXu podobnih operacijskih sistemov, skupaj z MacOS, BSD, Solaris in njihovimi izvedenkami.

SUS oz. Enotna UNIX specifikacija \cite{SUS_2020} je delo podjetja X/Open. Namen je poenotiti vse UNIX-u podobne sisteme in zagotavljati medsebojno kompatibilnost in prenosljivost kode.
Sestavlja jo pet dokumentov:
\begin{itemize}
	\item Definicija sistemskih vmesnikov,
	\item Sistemski vmesniki in zaglavne datoteke,
	\item Ukazi in podporni programi,
	\item Omrežne storitve in
	\item X/Open Curses.
\end{itemize}

Jedro SUS pa je POSIX oz. Prenosni Vmesnik za Operacijske Sisteme \cite{POSIX.1_2024}, ki ga vzdržuje delovna skupina Austin Group s člani iz organizacij ISO, IEEE in The Open Group.
Standard je objavljen pod oznako IEEE 1003 in ISO/IEC 9945.
Sestavljata ga dva dela: POSIX.1, ki vsebuje dejanske definicije sistemskih vmesnikov in storitev, in POSIX Testiranje Ustreznosti, ki definira teste za določanje konformnosti s POSIX.1.

POSIX.1 standard, in posledično SUS, se osredotoča na prenosnost aplikacij med UNIX-podobnimi sistemi, definiranje vmesnika jedra in minimalno definicijo vmesnika.
Ta standard je osnova za vse UNIX-podobne sisteme in omogoča enotno izkušnjo tako razvijalcem kot tudi uporabnikom.

\subsection{Zgodovina}

Razvoj jedra Linux se je začel leta 1991, ko je Finski študent Linus Torvalds začel razvijati majhno jedro za procesor 80386 -- prvi 32-bitni procesor v Intelovem naboru.

Zgodaj v razvoju je bila Linux izvorna koda javno objavljena in je kmalu pritegnila veliko skupnost programerjev, ki so prispevali k razvoju jedra in programske opreme.
Sistem je hitro zrasel iz majhne nepopolne replike UNIX sistema v popolnoma funkcionalen moderen operacijski sistem.

Prvo Linux jedro je bilo javno izdano 14. maja 1991. Ni podpiralo omrežij in delovalo je le na procesorjih Intel 80386.
Podpora za strojno opremo je bila zelo omejena in podprt je bil samo datotečni sistem Minix, ki je bil originalno razvit kot del učnega operacijskega sistema Minix.

Naslednji mejnik, Linux 1.0, je bil izdan 14. marca 1994. Dodana je bila podpora za TCP/IP omrežne protokole in BSD kompatibilna implementacija vtičnic (\textit{angl. socket}).
Drastično je bila izboljšana tudi podpora za strojno opremo, dodana je bila podpora disketne pogone, CD-ROM naprave, zvočne kartice, miške in mednarodne tipkovnice.
Prav tako je bila dodana podpora za več procesorjev, vendar še vedno omejena na Intel procesorje.
Omembe vredni so še podpora za deljen pomnilnik, semaforje, sporočilne vrste in medprocesno komunikacijo.

V sledečih verzijah je bila sčasoma dodana podpora za nove arhitekture procesorjev, kot so DEC Alpha, Sun SPARC in MIPS ter kasneje še PowerPC, ARM in mnoge druge.
%\cite{Silberschatz_Galvin_Gagne_2018}

\subsection{Arhitektura}

Linux sistem si lahko predstavljamo kot neke vrste piramido, kot ilustrira slika \ref{fig:linux_architecture}.
Na dnu piramide je strojna oprema, ki jo nadzira naslednji sloj -- operacijski sistem.
Nad operacijskim sistemom pa imamo sloja standardne knjižnice (\textit{angl. standard library}) in sistemskih ter uporabniških programov.

\begin{figure}[h!]
	\begin{center}
		\includegraphics[width=0.9\textwidth]{images/linux_layers.png}
	\end{center}
	\caption{Arhitektura Linux OS \cite{Tanenbaum_Bos_2023}}
	\label{fig:linux_architecture}
\end{figure}

Kar pa nas tu zares zanima, so vmesniki med posameznimi plastmi.
Predvsem nas zanima vmesnik sistemskih klicev, ki jih ponuja jedro, in vmesnik standardne knjižnice, ki nam olajša klicanje sistemskih klicev.

Standardna knjižnica nas zanima predvsem ker je uporaba vmesnika sistemskih klicev zelo odvisna od arhitekture procesorja.
Prav tako C in drugi višje nivojski programski jeziki nimajo podpore za klicanje specifičnih ukazov procesorja, ki jih potrebujemo za klicanje sistemskih klicev, to je mogoče zgolj preko zbirnega jezika \cite{Tanenbaum_Bos_2023}.
Zato bomo v nadaljevanju uporabljali standardno knjižnico, ki izpostavi preproste funkcije in poskrbi za same klice v jedro.

Ker POSIX standardi definirajo funkcije, njihove parametre, funkcionalnosti in rezultate standardne knjižnice, bo naša koda tehnično izvedljiva na katerem koli sistemu skladnem s POSIX standardom.

\subsection{Struktura jedra}

Jedro deluje direktno nad strojno opremo in ga je mogoče logično razdeliti v nekaj komponent, kot prikazuje slika \ref{fig:linux_kernel_architecture}.

\begin{figure}[h!]
	\begin{center}
		\includegraphics[width=0.9\textwidth]{images/linux_kernel_structure.png}
	\end{center}
	\caption{Arhitektura jedra Linux \cite{Tanenbaum_Bos_2023}}
	\label{fig:linux_kernel_architecture}
\end{figure}

Na najnižjem nivoju jedro implementira prekinitvene rutine (\textit{angl. interrupt handler}), ki omogočajo interakcijo z napravami, 
in dodeljevalnik (\textit{angl. dispatcher}), ki upravlja procese v izvajanju in omogoča menjavanje konteksta izvajanja.

Na višjem nivoju ločujemo razne jedrne podsisteme v tri komponente.
Vhodno-izhodna komponenta vsebuje vse sisteme za interakcijo z napravami sistema \cite{Tanenbaum_Bos_2023}.
Na najvišjem nivoju komponenta združuje vse V/I operacije pod navideznim datotečnim sistemom oz. VFS (\textit{angl. Virtual File System}), ki si ga bolj podrobno pogledamo v poglavju \ref{ssec:linux:vfs}.
Na najnižjem nivoju pa se vse V/I operacije izvedejo v nekem gonilniku naprave.
Linux gonilnike klasificira kot gonilnike znakovnih naprav, ki ne omogočajo naključnega dostopa, in gonilnike bločnih naprav, ki omogočajo naključni dostop.
V diagramu so gonilniki omrežnih naprav zapisani posebej, ker so implementirani nekoliko drugače, vendar so iz tehničnega vidika klasificirani kot znakovne naprave.

Drugi dve ključni komponenti, na sliki \ref{fig:linux_kernel_architecture}, sta komponenta za upravljanje pomnilnika in komponenta za upravljanje procesov \cite{Tanenbaum_Bos_2023}.
Komponenta za upravljanje pomnilnika skrbi za povezavo virtualnega in fizičnega pomnilnika, povezane strani v pomnilniku, predpomnilnik in prenašanje novih strani v pomnilnik.
Komponenta za upravljanje procesov pa skrbi za življenjske cikle in razvrščanje procesov.
% Note: Linux obravnava tako procese kot tudi niti kot izvajalne entitete

\subsubsection{Linux Virtual File System} \label{ssec:linux:vfs}

Navidezni datotečni sistem (\textit{angl. Virtual File System -- VFS}) je programska plast v jedru, ki omogoča uporabnikom in uporabniškim programom dostop do različnih datotečnih sistemov.
Deluje pa tudi kot abstrakcija znotraj jedra, ki omogoča sobivanje različnih datotečnih sistemov.

VFS \cite{Linux_kernel_docs_LVFS} implementira vse funkcionalnosti, ki jih pričakujemo od datotečnega sistema, kot so branje in pisanje datotek, ustvarjanje in brisanje datotek, itd.
Pot do imenika, ki je podana omenjenim ukazom, se v VFS preko predpomnilnika imeniških vnosov (\textit{angl. dentry cache / dcache}) pretvori v imeniški vnos (\textit{angl. dentry}).
Ta mehanizem omogoča zelo hitro prevedbo poti v imeniški vnos, saj se izognemo branju iz diska, če seveda je imeniški vnos na voljo v predpomnilniku.

Vsak imeniški vnos vsebuje kazalec na indeksno vozlišče (\textit{angl. inode}). Indeksna vozlišča so objekti datotečnega sistema (npr. datoteke, imeniki, naprave), ki bivajo na disku (za datotečne sisteme blokovnih naprav) ali v pomnilniku (za psevdo datotečne sisteme).
Indeksna vozlišča, ki bivajo na disku se, po potrebi, kopirajo v pomnilnik, spremembe pa se zapišejo nazaj na disk.
Eno indeksno vozlišče lahko pripada več imeniškim vnosom, kar omogoča, da se datoteka prikaže na več različnih mestih v datotečnem sistemu.

VFS omogoča UNIX koncept ``everything is a file'' \cite{Garrels_2008} oz. bolj specifično ``everything is a file descriptor'' \cite{LWN_Brown_2010}, kjer je osnovna ideja, da se vsi sistemski resursi (npr. naprave, procesi, omrežje) predstavijo kot datoteke.
To omogoča enoten način dostopa do sistemskih virov, kar olajša uporabo in razvoj programske opreme in zmanjša število potrebnih sistemskih klicev.

\section{Operacijski sistem Windows}

Windows je zaprtokodni operacijski sistem, ki ga razvija Microsoft.
Nastopa v več različnih družinah, vsaka s svojo ciljno publiko od uporabniških računalnikov in strežnikov do IoT in vgrajenih sistemov.

Danes je Windows najbolj popularen operacijski sistem za namizne računalnike, s približno 70\% tržnim deležem.

\subsection{Zgodovina}

Razvoj Windows se je začel leta 1988 pod okriljem Microsoft in IBM.
Cilj je bil razviti NT (\textit{``New Technology''}) operacijski sistem s podporo za OS/2 in POSIX aplikacijska vmesnika, vendar je tekom razvoja nastal ločen 32-bitni Windows aplikacijski vmesnik pod imenom Win32 \cite{Silberschatz_Galvin_Gagne_2018}.

V NT verziji 4.0, ki se je uporabljala v Windows 95, je jedro vsebovalo spletni strežnik in brskalnik, skupaj z vso grafično logiko in uporabniškim vmesnikom.
Te spremembe so bile uvedene z namenom izboljšanja učinkovitosti, stranski učinek pa je bila zmanjšana zanesljivost sistema.

V Windows 2000 so dodali podporo za Active Directory, ``plug-and-play'' naprave in porazdeljen datotečni sistem.
Prav tako so izboljšali omrežno podporo ter podprli več različnih procesorjev, vendar samo Intel (in kompatibilne) procesorje, in več sistemskega pomnilnika.

Naslednji večji skok v razvoju je bil Windows XP, ki je na trg nastopil z prenovljenim grafičnim okoljem in samodejnim odpravljanjem težav z aplikacijami in samim operacijskim sistemom.
S temi spremembami je Windows XP zagotavljal boljše omrežne storitve, podporo za več naprav ter ogromno izboljšavo v zmogljivosti, zanesljivosti in varnosti.
Prav tako je bila prva izdaja s podporo za 64-bitne sisteme.

Leta 2006 je bil izdan Windows Vista, ki kljub mnogim izboljšavam, zaradi težav z nekompatibilnostmi in počasnim odzivnim časom, ni požel velikega uspeha.
Med najbolj ključnimi izboljšavami je bil premik grafične implementacije v ``desktop window manager'', ki se je izvajal kot uporabniški proces.

Naslednji večji skok je nastopil z Windows 7, ki je bil izdan leta 2009, in je odpravil večino težav svojega predhodnika.
% TODO: Win 8, 10, 11
%\cite{Silberschatz_Galvin_Gagne_2018}

Glavna cilji pri razvoju Windows NT so bili \cite{Yosifovich_Russinovich_Solomon_Ionescu_2017}:
\begin{description}
	\item[Razširljivost] Koda mora biti napisana tako, da lahko brez težav raste in se prilagodi zahtevam na trgu.
	\item[Prenosljivost] Sistem se mora izvajati na različnih arhitekturah in ohranjati možnost prehoda na nove arhitekture.
	\item[Zanesljivost in robustnost] Sistem se mora zaščititi pred internimi napakami in zunanjimi vplivi. Aplikacije ne morejo škodovati OS ali drugim aplikacijam.
	\item[Kompatibilnost] Kljub temu, da Windows NT razširja funkcionalnosti obstoječih tehnologij, naj bodo programski vmesniki kompatibilni s starejšimi verzijami Windows in MS-DOS. Prav tako naj dobro sodeluje z drugimi sistemi kot UNIX, OS/2 ub NetWare.
	\item[Zmogljivost] Upoštevajoč druge zahteve, naj bo sistem kar se da zmogljiv in hiter na vsaki platformi.
\end{description}

Seveda so se te cilji spreminjali tekom razvoja.

\subsection{Arhitektura} \label{ssec:windows:architecture}

Windows je monoliten operacijski sistem, v smislu da si večina kode OS in gonilnikov deli isti privilegiran naslovni prostor \cite{Yosifovich_Russinovich_Solomon_Ionescu_2017}.
To pomeni, da lahko katera koli komponenta OS ali gonilnik potencialno okvari/spremeni podatke, ki jih uporabljajo druge komponente OS.

V poskusu zmanjšanja verjetnosti tovrstnih napak Windows zahteva digitalno podpisane gonilnike.
Gonilnik mora najprej z ustreznim certifikatom podpisati izdajatelj, ter ga oddati v pregled na Microsoftov portal.
Ko Microsoft potrdi delovanje gonilnika, ga podpiše še s svojim digitalnim certifikatom in šele nato bo operacijski sistem sprejel in izvajal gonilnik \cite{Yosifovich_Russinovich_Solomon_Ionescu_2017}.

\begin{figure}[h!]
	\begin{center}
		\includegraphics[width=\textwidth]{images/windows_programming_layers.png}
	\end{center}
	\caption{Arhitektura Windows \cite{Tanenbaum_Bos_2023}}
	\label{fig:windows_architecture}
\end{figure}

Jedro operacijskega sistema NT je jedrni program NTOS (\textit{ntoskrnl.exe}), ki izpostavlja vmesnik sistemskih klicev.
Vmesnik sistemskih klicev in uporabniška knjižnica (\textit{ntdll.dll}) sta v celoti delo Microsoft programerjev in v večjem delu nedokumentirana.
Uporabniški vmesniki so del osebnosti operacijskega sistema, ki se imenujejo podsistemi (\textit{angl. subsystem}).

Originalno je NT podpiral tri podsisteme: Win32, POSIX in OS/2.
Podpora za OS/2 je bila odstranjena v izdaji Windows XP, podpora za POSIX pa v Windows 8.1.
Danes so vse Windows aplikacije napisane z uporabo Win32 API, ki ga izpostavlja Win32 podsistem.
Microsoft preko Github projekta \href{https://github.com/microsoft/win32metadata}{microsoft/win32metadata} objavlja opis celotne Win32 API površine v standardiziranem formatu (ECMA-335) na tak način, da omogoča izgradnjo projekcij v več programskih jezikov.
To omogoča pisanje Windows aplikacij tako v privzetem C/C++ kot tudi v drugih jezikih.

\subsection{Windows Podsistemi}

Kot prikazuje slika \ref{fig:windows_subsystems_components} so NT podsistemi sestavljeni iz štirih komponent: procesa podsistema, nabora knjižnic, sistemskih kljuk in podpore v jedru \cite{Tanenbaum_Bos_2023}.
Proces podsistema je storitev, ki jo zažene upravljalnik sej (\textit{smss.exe}) -- prvi uporabniški proces, ki se zažene v Windows.

Kljub temu, da je Win32 sedaj edini podprti podsistem, se Windows drži modela podsistemov za potencialne prihodnje razširitve.

\begin{figure}[h!]
	\begin{center}
		\includegraphics[width=0.9\textwidth]{images/windows_subsystems_components.png}
	\end{center}
	\caption{Komponente Windows podsistemov \cite{Tanenbaum_Bos_2023}}
	\label{fig:windows_subsystems_components}
\end{figure}

\subsubsection{Win32 podsistem}

Win32 funkcijski klici se kolektivno imenujejo Win32 API.
To so polno dokumentirani javno objavljeni vmesniki, implementirani kot procedure, ki kličejo NT sistemske klice ali, v nekaterih primerih, opravijo delo v uporabniškem načinu \cite{Tanenbaum_Bos_2023}.
Za ohranjanje aplikacijske kompatibilnosti se obstoječi Win32 API klici ne spreminjajo z novimi izdajami Windows, se pa dodajajo nove funkcije.

Programi napisani za starejše verzije Win32, bi teoretično morale delovati v novejših verzijah sistema, saj se je Win32 API zelo malo spreminjal.
V imenu kompatibilnosti Windows podpira dve posebni izvajalni okolji imenovani WoW oz. Windows-on-Windows:
\begin{description}
	\item[WoW32] se uporablja na 32-bitnih x86 sistemih za izvajanje 16-bitnih Windows aplikacij. Zadnja izdaja sistema, ki podpira WoW32 je bil Windows 10.
	\item[WoW64] pa omogoča izvajanje 32-bitnih aplikacij na 64-bitnem sistemu.
	V Windows 10 je bila ta funkcionalnost še razširjena za izvajanje 32-bitnih x86 aplikacij na arm64 sistemih.
	V Windows 11 pa ta emulacija podpira tudi izvajanje 64-bitnih x86 aplikacij na arm64 sistemih.
\end{description}
V Windows 10 je bila ta funkcionalnost še razširjena za izvajanje 32-bitnih x86 aplikacij na arm64 sistemih.
V Windows 11 pa ta emulacija podpira tudi izvajanje 64-bitnih x86-64 aplikacij na arm64 sistemih.

Osnovna filozofija Win32 se precej razlikuje od UNIX pristopa.
V UNIX sistemih so funkcije OS precej preproste, z malo parametri in redko najdemo več kot en način za izvedbo določene operacije.
Win32 pa zagotavlja obsežen nabor vmesnikov, z veliko parametri in običajno najdemo tri ali štiri načine za izvedbo določene operacije.
Pogosto vidimo tudi mešanje nizko in visoko nivojskih funkcij v istem vmesniku.
To pomeni, da Win32 zagotavlja funkcionalno zelo bogat nabor vmesnikov, vendar vnese veliko kompleksnosti zaradi slabše strukture vmesnikov, ki mešajo nizko in visoko nivojske funkcije v isti API \cite{Tanenbaum_Bos_2023}.

\subsection{Register Windows}

Ob zagonu sistema Windows ustvari imenski prostor NT, na katerega se pripenjajo datotečni volumni.
To pomeni, da mora sistem nekje drugje pridobiti konfiguracijo potrebno ob zagonu.
V ta namen se uporablja poseben datotečni sistem, ki se pripne na imenski prostor NT.
Ta datotečni sistem se imenuje register (\textit{angl. registry}) in je optimiziran za majhne datoteke \cite{Tanenbaum_Bos_2023}.

Register je organiziran v ločene volumne, ki jih imenujemo gnezda (\textit{angl. hive}).
Vsako gnezdo se hrani v svoji datoteki v imeniku \texttt{C:\symbol{92}Windows\symbol{92}system32\symbol{92}config\symbol{92}} na zagonskem volumnu.

Ob zagonu sistema se gnezdo \texttt{SYSTEM} naloži v pomnilnik.
V njem se hranijo ključne informacije kot gonilniki, zagonski programi in drugi parametri za inicializacijo sistema.

Poleg tega ima Windows še druga gnezda kot vidimo v tabeli \ref{tab:windows_registry_hives}.

\begin{table}[h!]
	\begin{center}
		\begin{tabular}{ p{3.7cm}|p{8.8cm} }
			Gnezdo     & Namen                                         \\
			\hline
			SYSTEM     & OS konfiguracija                              \\
			HARDWARE   & zapisi o zaznani strojni opremi               \\
			BCD        & baza za konfiguracijo obuvanja                \\
			SAM        & informacije o lokalnih uporabniških računih \\
			SECURITY   & informacij o zaščiti                        \\
			DEFAULT    & privzeto gnezdo za nove uporabnike            \\
			NTUSER.DAT & gnezdo uporabnika, v domačem imeniku         \\
			SOFTWARE   & aplikacijski razredi COM                      \\
			COMPONENTS & manifesti in odvisnosti sistemskih komponent  \\
		\end{tabular}
	\end{center}
	\caption{Gnezda registrov Windows \cite{Tanenbaum_Bos_2023}}
	\label{tab:windows_registry_hives}
\end{table}

\section{Sistemski klici} \label{sec:syscalls}

Sistemski klici so vmesnik do storitev, ki jih ponuja jedro operacijskega sistema, kot na primer ustvarjanje procesov, branje in pisanje datotek, komunikacijo med procesi, itd.
Ker torej sistemski klici prehajajo med uporabniškim in jedrnim načinom procesorja (več o tem v poglavju \ref{sec:syscall_execution}), je njihova implementacija odvisna od arhitekture procesorja in njegovega nabora ukazov.

Ker pa jih uporabniški programi potrebujejo dokaj pogosto, so v večini operacijskih sistemov izpostavljeni preko knjižnice ali API-ja.
Najbolj pogosto uporabljena API-ja za aplikacijske programerje sta \textbf{POSIX}, za sisteme ki sledijo POSIX standardu (npr. Unix, Linux in macOS), in \textbf{Windows API}, za Windows sisteme.
Programer do API dostopa preko knjižnice, ki jo ponuja operacijski sistem -- npr. glibc za programski jezik C, v primeru Linuxa.
Tu je omembe vredno tudi, da imena funkcij v knjižnici niso nujno enaka imenom sistemskih klicev, ki jih uporablja sistem.

Ker je sistemskih klicev ogromno, jih bomo v nadaljevanju razdelili na šest kategorij \cite{Silberschatz_Galvin_Gagne_2018}, in sicer:
\begin{description}
	\item[upravljanje procesov] kreiranje in ustavljanje procesov, nalaganje in izvajanje programa, upravljanje atributov procesa, sprejemanje in oddajanje signalov, zasedanje in sproščanje pomnilnika
	\item[upravljanje datotek] kreiranje, brisanje, odpiranje in zapiranje datotek, branje in pisanje, premikanje datotek, upravljanje atributov datotek
	\item[upravljanje naprav] zahtevanje in sproščanje naprav, branje in pisanje v napravo, upravljanje atributov naprav, logično pripenjanje in odpenjanje naprav
	\item[vzdrževanje informacij o sistemu] upravljanje sistemskega časa, upravljanje informacij sistema, pridobivanje in nastavljanje atributov procesov, datotek in naprav
	\item[komunikacija med procesi] kreiranje in brisanje komunikacijskih kanalov, pošiljanje in sprejemanje sporočil, prenos informacij o statusu, pripenjanje in odpenjanje oddaljenih naprav
	\item[zaščita] upravljanje pravic procesov, datotek in naprav
\end{description}

% Preden pa se podamo v primerjave posameznih sistemskih klicev, si poglejmo še, kakšne so implementacijske razlike med sistemskimi klici v Linuxu in Windowsu.
% Tu še enkrat poudarjam, da so razlike med sistemskimi klici odvisne od arhitekture procesorja.
% V vseh konkretnih primerih bom uporabljal ukazni nabor in imena registrov arhitekture \textbf{x86}.

\section{Izvedba sistemskega klica} \label{sec:syscall_execution}

Kot smo že povedali, imajo operacijski sistemi dve glavni funkciji: upravljanje sistemskih virov in zagotavljanje abstrakcij za uporabniške aplikacije.
Upravljanje virov je v večini primerov transparentno uporabnikom in se izvaja v ozadju. Zato je vmesnik med programi in operacijskim sistemom predvsem osredotočen na abstrakcije sistema \cite{Tanenbaum_Bos_2023}.
Da lahko res razumemo kaj operacijski sistemi počnejo, si moramo pobližje pogledati ta vmesnik.

Moderne procesorske arhitekture definirajo več privilegijskih nivojev ali obročev.
Vzemimo za primer arhitekturo x86 \cite{Intel_2024}, ki implementira 4 nivoje oz. obroče (številčene od 0 do 3) kot prikazuje slika \ref{fig:privilege_levels_x86}.
Večja številka pomeni manj privilegijev.
Srednji obroč oz. nivo 0 se uporablja za najbolj kritično programsko opremo, običajno jedro operacijskega sistema.
Zunanji obroči oz. nivoji 1 do 3 pa se uporabljajo za manj kritično programsko opremo.
Kako operacijski sistem koristi nivoje izvajanja je odvisno od implementacije.
Sistemi, ki uporabljajo samo dva nivoja izvajanja, uporabljajo nivo 0 za jedro in nivo 3 za uporabniške programe.

Procesor uporabi nivoje, da zaščiti segmente bolj privilegiranih procesov pred manj privilegiranimi.
Prav tako ima procesor poseben nabor navodil, ki se imenujejo privilegirana navodila.
Le-ta se lahko uporabljajo samo v nivoju 0.

Kadar procesor zazna kršitev nivoja, sproži izjemo in tako obvesti jedro o napaki.

\begin{figure}[h!]
	\begin{center}
		\begin{tikzpicture}
			\draw (0, 0) circle (1);
			\draw (0, 0) circle (2);
			\draw (0, 0) circle (3);
			\draw (0, 0) circle (4);
			\node [align=right] at (-6.5, -1.5) {
				\shortstack[r]{\small{Operating}\\\small{System}\\\small{Kernel}}\\\\
				\shortstack[r]{\small{Operating System}\\\small{Services}}\\\\
				\small{Applications}
			};
			\node at (0, -.2) {Level 0};
			\node at (0, -1.5) {Level 1};
			\node at (0, -2.5) {Level 2};
			\node at (0, -3.5) {Level 3};
			\draw[->] (-4.5, .3) -- (0, .3);
			\draw[->] (-4.5, -1.7) -- (0, -1.2);
			\draw[->] (-4.5, -1.7) -- (0, -2.2);
			\draw[->] (-4.5, -3.2) -- (0, -3.2);
		\end{tikzpicture}
	\end{center}
	\caption{Nivoji zaščite v x86 arhitekturi \cite{Intel_2024}}
	\label{fig:privilege_levels_x86}
\end{figure}

Ker pa je število nivojev in njihov namen drugačen od arhitekture do arhitekture, jih posplošimo dva nivoja delovanja: \textbf{uporabniški} in \textbf{jedrni} oz. privilegiran način.
Kot že omenjeno, se jedro operacijskega sistema izvaja v jedrnem načinu, kjer ima dostop do vseh virov sistema in lahko izvede kateri koli ukaz v ukaznem naboru procesorja.
Preostanek programske opreme pa se izvaja v uporabniškem načinu, kjer ima omejen dostop.

Ker je način izvedbe klica odvisen od arhitekture sistema in običajno zahteva implementacijo v zbirnem jeziku, večina operacijskih sistemov zagotavlja knjižnico v višje-nivojskem jeziku -- običajno C ali C++.
Kljub temu, da iz uporabniške oz. programske strani sistemski klici v knjižnici izgledajo kot običajne funkcije, pa je njihova implementacija popolnoma drugačna \cite{Tanenbaum_Bos_2023}:
\begin{enumerate}
	\item Uporabniški proces (oz. knjižnica, ki jo uporablja) shrani številko sistemskega klica in argumente funkcije v registre procesorja -- registri se razlikujejo glede na arhitekturo in OS.
	\item Uporabniški proces proži past, ki preklopi procesor v jedrni način in požene prekinitveni servisni program (\textit{angl. Interrupt Service Routine -- ISR}), ki ga je definiral operacijski sistem.
	\item Jedro izvede zahtevan sistemski klic kot katero koli drugo funkcijo in vrne rezultat -- rezultat se zapiše v register iz katerega ga uporabniški program lahko prebere.
	\item Procesor preklopi nazaj v uporabniški način in nadaljuje izvajanje uporabniškega procesa.
\end{enumerate}

\section{Programski vmesnik Linux -- Linux API}

Linux API je implementiran preko glibc oz. GNU C knjižnice, ki zagotavlja API vmesnik po standardih ISO C11, POSIX.1-2008 in BSD ter druge OS-specifične API vmesnike \cite{GNU_Manual}.
Prav tako je kompatibilna s starejšimi verzijami ISO C.
Celotna GNU C knjižnica je odprtokodna pod licenčnimi pogoji LGPL 2.1.

Ker pa je Linux jedro odprtokodno pod licenco GPL 2.0, si lahko pogledamo tudi celotno implementacijo jedra.
Za nas je najbolj relevantna tabela sistemskih klicev.
Ta je, za x86 64-bitno arhitekturo dostopna v datoteki \texttt{arch/x86} \texttt{/entry/syscalls/syscall\_64.tbl} v \href{https://github.com/torvalds/linux}{izvorni kodi jedra}.
Tu lahko opazimo, da je arhitektura procesorja del poti do datoteke in če pogledamo v druge imenike v \texttt{arch}, lahko vidimo, da ima vsaka implementirana arhitektura med drugim tudi svojo tabelo sistemskih klicev.

Kot smo že povedali Linux sledi standardu POSIX, ki definira procedure aplikacijskega vmesnika.
V veliki večini te procedure zahtevajo uporabo sistemskih klicev, vendar je potrebno poudariti, da povezava ni nujno ena proti ena.
Lahko se zgodi, da je neka procedure izvedljiva v uporabniškem načinu in ne zahteva sistemskega klica, ali pa zahteva uporabo več sistemskih klicev.
V nekaterih primerih se lahko zgodi tudi, da več procedur uporablja isti (bolj splošen) sistemski klic.
Te specifike si bomo pobližje ogledali v naslednjem poglavju.

\section{Programski vmesnik Windows -- Windows API}

Windows je, v kontrastu z Linuxom, zaprtokodni operacijski sistem, zato je težje najti dokumentacijo o njegovi implementaciji.
Ena izmed glavnih razlik je, da Windows omogoča dostop do sistemskih klicev izključno preko Windows API-ja.
Torej, če smo v Linuxu lahko pokukali v tabelo sistemskih klicev in jih lahko celo poklicali direktno z zbirnim jezikom, je v Windows to zahtevno in manj zanesljivo.

Še ena velika razlika, ki jo bomo srečavali v naslednjem poglavju, je število API procedur.
Windows API izpostavi ogromno procedur, v rangu več tisoč, medtem ko POSIX API izpostavi le nekaj sto \cite{Tanenbaum_Bos_2023}.
To je posledica več faktorjev -- pogosto več funkcij uporablja isti sistemski klic, veliko pa je tudi funkcij, ki so v celoti implementirane v uporabniškem načinu.

Ker Windows ne izpostavi sistemskih klicev direktno, se lahko ti spreminjajo med posameznimi verzijami operacijskega sistema \cite{Tanenbaum_Bos_2023}.
To pomeni, da je težko zagotovo reči ali je neka funkcija implementirana v jedru ali uporabniški knjižnici.

Windows se od večine operacijskih sistemov razlikuje po kodiranju internih znakovnih nizov \cite{Yosifovich_Russinovich_Solomon_Ionescu_2017}.
Uporablja namreč 16-bitno Unicode kodiranje imenovano UTF-16.
Starejše izdaje sistema pa so uporabljale 8-bitno Windows-1252 kodiranje, ki je bilo zasnovano na podlagi osnutka za standard ANSI (\textit{American National Standards Institute}).
V strokovni literaturi in celo Windows dokumentaciji zato pogosto zasledimo napačno poimenovanje Windows-1252 kodiranja kot ``ANSI kodiranje.''
Ker mnogo starejših aplikacij še vedno uporablja Windows-1252 kodiranje, ima večina Windows funkcij dve verziji: tako, ki sprejema Windows-1252 znake in tako, ki sprejema UTF-16 znake.
Windows-1252 verzije funkcij so posledično rahlo počasnejše, saj se niz najprej pretvori v UTF-16.

V poglavju \ref{ssec:windows:architecture} smo omenili, da jedrni program NTOS (\textit{ntoskrnl.exe}) implementira in izpostavlja vmesnik sistemskih klicev, ki jih potem iz uporabniške strani Windows API kliče preko uporabniške knjižnice \textbf{ntdll.dll}.
Kolektivno \textit{ntoskrnl.exe} in \textit{ntdll.dll} predstavljata NT API, ki je v večini nedokumentiran.
Prav tako uporaba teh sistemskih klicev ni priporočena, saj Windows zagotavlja konsistenco in stabilnost na nivoju WinAPI vmesnika.
Seveda pa to ni ustavilo številnih raziskovalcev, da bi pokukali v globine sistema in spisali nekaj neuradnih dokumentacij NT vmesnika:
\begin{itemize}
	\item \href{http://undocumented.ntinternals.net/}{Dokumentacija funkcij NT API}
	\item \href{https://j00ru.vexillium.org/syscalls/nt/64/}{Tabela sistemskih klicev Windows x86-64}
\end{itemize}

\begin{table}[h!]
	\begin{center}
		\begin{tabular}{ l|l }
			Win32 API klic         & NT API klic                 \\
			\hline
			\verb|CreateProcess|   & \verb|NtCreateProcess|      \\
			\verb|CreateThread|    & \verb|NtCreateThread|       \\
			\verb|ReadFile|        & \verb|NtReadFile|           \\
			\verb|DeleteFile|      & \verb|NtSetInformationFile| \\
			\verb|DuplicateHandle| & \verb|NtDuplicateObject|    \\
			\verb|CloseHandle|     & \verb|NtClose|              \\
		\end{tabular}
	\end{center}
	\caption{Primeri Win32 API klicev in vezanih NT API klicev \cite{Tanenbaum_Bos_2023}}
	\label{tab:example_win32_nt_mapping}
\end{table}

V tabeli \ref{tab:example_win32_nt_mapping} si poglejmo nekaj primerov Win32 API klicev in NT API klicev, ki jih kličejo.
Zanimivo je, da so povezave tako nezanimive.
Večina nizko nivojskih Win32 funkcij ima direktne NT ekvivalente kar ni preveč presenetljivo, saj je bil Win32 oblikovan z NT v mislih \cite{Tanenbaum_Bos_2023}.
Pogosto Win32 funkcije samo manipulirajo parametre in kličejo NT funkcije.
Nekateri Win32 klici sprejmejo pot do objekta medtem ko NT ekvivalentna funkcija pričakuje ročico objekta (\textit{angl. handle}).
Prav tako je naloga Win32 API klicev, da pretvorijo znakovne nize iz ANSI v Unicode kodiranje.

% TODO: Nedokumentiran del API-ja (Ntoskrnl.exe, Win32k.sys, ntdll.dll) pg. 49, Ne-strukturirano procesno drevo

\chapter{Procesi}

Koncept procesa je ključen za večopravilne operacijske sisteme.
Proces definiramo kot program (strojna koda) v izvajanju.
Če na primer 16 ljudi sočasno izvaja nek program to pomeni, da se izvaja 16 procesov, ki izvajajo isto strojno kodo \cite{Bovet_Cesati_2005}.

V tem poglavju bomo spoznali kaj so procesi, kako so ustvarjeni in končani ter kakšne so razlike med njihovo implementacijo v Linux in Windows.

\section{Proces}

Izraz ``proces'' se pogosto uporablja z različnimi pomeni \cite{Bovet_Cesati_2005}, mi ga bomo opredelili kot program (strojna koda) v izvajanju.
Lahko jih obravnavamo kot zbirka podatkovnih struktur, ki v celoti opisujejo izvajanje programa \cite{Bovet_Cesati_2005}.
Iz perspektive jedra so procesi entiteta, ki zaseda sistemske vire (CPU čas, pomnilnik, \dots).

Procesi so v mnogih pomenih podobni živim bitjem -- so ustvarjeni, imajo bolj ali manj pomenljivo življenje, lahko ustvarijo enega ali več otroških procesov ter na koncu umrejo oz. se zaključijo.

Jedro operacijskega sistema beleži status procesa preko jedrnih podatkovnih struktur, ki med drugim vsebujejo:
\begin{itemize}
	\item identifikator procesa in starša,
	\item stanje (življenjski cikel),
	\item programski števec,
	\item kazalnik na sklad,
	\item pomnilniške dodelitve,
	\item tabelo odprtih sistemskih objektov (datoteke, ključavnice, semaforji \dots),
	\item informacije o izvajanju (uporabniški čas, jedrni čas, \dots),
	\item informacije o razvrščanju (\textit{angl. scheduling}) \dots
\end{itemize}
Več o teh jedrnih strukturah si bomo pogledali pri vsakem sistemu posebej, saj se implementacije razlikujejo.

Vsak proces ima svoj prostor v pomnilniku, ki je običajno razdeljen na več segmentov kot je ilustrirano na sliki \ref{fig:process_memory_segments}:
\begin{itemize}
	\item \textbf{podatkovni} (\textit{angl. data}) -- vsebuje vrednosti globalnih spremenljivk,
	\item \textbf{programski} (\textit{angl. text}) -- vsebuje strojno kodo programa,
	\item \textbf{sklad} (\textit{angl. stack}) -- vsebuje vrednosti lokalnih spremenljivk in povratne naslove funkcij in
	\item \textbf{kopica} (\textit{angl. heap}) -- dinamično dodeljen pomnilnik \cite{Silberschatz_Galvin_Gagne_2018}.
\end{itemize}

\begin{figure}[h!]
	\begin{center}
		\begin{tikzpicture}
			\filldraw[fill=blue!20!white] (0, 0) rectangle (4, 8);
			\draw node at (-1, 0) {0};
			\draw node at (-1, 8) {max};
			\filldraw[fill=gray!40!white] (0, 8) rectangle (4, 6.5) node[pos=.5] {stack};
			\draw[arrow] (2, 6.5) -- (2, 5.8);
			\draw[arrow] (2, 3.5) -- (2, 4.2);
			\filldraw[fill=gray!40!white] (0, 2) rectangle (4, 3.5) node[pos=.5] {heap};
			\filldraw[fill=gray!40!white] (0, 1) rectangle (4, 2) node[pos=.5] {data};
			\filldraw[fill=gray!40!white] (0, 0) rectangle (4, 1) node[pos=.5] {text};
		\end{tikzpicture}
	\end{center}
	\caption{Segmenti pomnilnika procesa \cite{Silberschatz_Galvin_Gagne_2018}}
	\label{fig:process_memory_segments}
\end{figure}

\section{Proces v Linux}

Novo ustvarjen Linux proces je skoraj identičen svojemu staršu.
Prejme lastno kopijo starševega naslovnega prostora in nadaljuje z izvajanjem iste kode takoj za sistemskim klicem, ki je ustvaril nov proces.
Če tudi pa si procesa delita kodo, imata ločeni kopiji podatkov (sklad in kopica), tako spremembe pomnilnika v otroku ne spremeni pomnilnika v starši in obratno.

Novejše verzije Linux jedra implementirajo tudi podporo za niti in večnitne procese preko t. i. lahkotnih procesov (\textit{Lightweight Process -- LWP}) \cite{Bovet_Cesati_2005}.
V osnovi si lahkotni procesi lahko delijo vire kot so pomnilniški prostor, odprte datoteke oz. datotečne opisnike ipd.
Istočasno pa omogočajo neodvisno razvrščanje za izvajanje, kar jih ločuje od uporabniške implementacije, ki se je uporabljala pred jedrno podporo.

Za združevanje niti Linux implementira koncept skupine niti, ki je množica lahkotnih procesov, ki se obnašajo kot celota v smislu sistemskih klicev \texttt{getpid()}, \texttt{kill()} in \texttt{exit()}.
% TODO: Premakni pod niti!

\subsection{Procesni deskriptor}

Kot smo že omenili, jedro beleži vse relevantne informacije procesa v posebni podatkovni strukturi.
V Linux sistemih se ta podatkovna struktura imenuje procesni deskriptor (\textit{angl. process descriptor}) -- struktura tipa \texttt{task\_struct}.
Procesni deskriptor poleg velikega števila atributov o procesu vsebuje tudi nekaj kazalnikov na druge podatkovne strukture, ki vsebujejo kazalnike na druge podatkovne strukture in tako dalje kot prikazuje slika \ref{fig:linux_process_descriptor}.

\begin{figure}[h!]
	\begin{center}
		\includegraphics[width=0.9\textwidth]{images/linux_process_descriptor.png}
	\end{center}
	\caption{Procesni deskriptor v Linux \cite{Bovet_Cesati_2005}}
	\label{fig:linux_process_descriptor}
\end{figure}

Celotna struktura \texttt{task\_struct} je precej velika (v mojem testnem okolju približno $9.5$ KiB) in vsebuje mnogo atributov.
Točna velikost in število atributov se spreminja z verzijo jedra, CPE arhitekturo sistema in atributi ob izgradnji jedra, zato si poglejmo samo najbolj relevantne:
\begin{description}
	\item[state] stanje (življenjski cikel) procesa, bolje opisano v poglavju \ref{ssec:linux_process:lifecycle}
	\item[pid] enolični numerični identifikator procesa, določen sekvenčno ob kreiranju; ta vrednost ima zgornjo mejo $32767$ in ko jo sistem doseže, prične reciklirati stare (sproščene) vrednosti
	\item[tgid] PID vodilnega procesa skupine niti
	\item[group\_leader] kazalnik na procesni deskriptor vodilnega procesa v skupini procesov
	\item[parent] kazalnik na procesni deskriptor procesa, ki prejme signal \texttt{SIGCHLD}
	\item[real\_parent] kazalnik na procesni deskriptor starša ali proces 1 (\textit{init}), če starš ne obstaja več
	\item[children] dvojni povezan seznam kazalnikov na procesne deskriptorje otroških procesov
	\item[sibling] dvojni povezan seznam kazalnikov na procesne deskriptorje sorodnih procesov oz. sorojencev
	\item[exit\_code] koda podana funkciji \texttt{exit()}
	\item[exit\_state] izhodno stanje procesa 
	\item[cred] poverilnice procesa
\end{description}

Ker so procesi dinamične entitete in lahko obstajajo vse od nekaj sekund do nekaj mesecev, se procesni deskriptorji hranijo v dinamičnem pomnilniku namesto statičnem jedrnem pomnilniku.
Vsak proces v tem dinamičnem pomnilniku hrani dve podatkovni strukturi -- majhno podatkovno strukturo (\texttt{thread\_info}), ki vsebuje kazalnik na procesni deskriptor in procesni sklad, ki se uporablja v jedrnem načinu.

Za naključne vpoglede in poizvedbe procesnih deskriptorjev Linux prav tako vzdržuje zgoščene tabele za polja \texttt{pid}, \texttt{tgid}, \texttt{pgrp} in \texttt{session} \cite{Bovet_Cesati_2005}.

\subsection{Življenjski cikel} \label{ssec:linux_process:lifecycle} % AKA stanje procesa

Omenili smo, da imajo procesi v Linuxu specifično polje \texttt{state}, ki nam pove v katerem stanju življenjskega cikla je proces.
Možna stanja za spremenljivko so \cite{Bovet_Cesati_2005}:
\begin{description}
	\item[\texttt{TASK\_RUNNING}] Proces se izvaja na CPE ali čaka na izvajanje.
	\item[\texttt{TASK\_INTERRUPTIBLE}] Proces je odložen (spi) dokler se ne izpolni željen pogoj (strojna prekinitev, sprostitev sistemskega vira, signal \dots).
	\item[\texttt{TASK\_UNINTERRUPTIBLE}] Enako kot \texttt{TASK\_INTERRUPTIBLE}, vendar procesa ne moramo zbuditi s signalom.
	To procesno stanje je redko uporabljeno \cite{Bovet_Cesati_2005}.
	\item[\texttt{TASK\_STOPPED}] Izvajanje procesa je zaustavljeno po prejemu signala (npr. \texttt{SIGSTOP}).
	\item[\texttt{TASK\_TRACED}] Izvajanje procesa je zaustavljeno preko razhroščevalnika.
	\item[\texttt{TASK\_ZOMBIE}] Proces se je končal, vendar starš še ni prevzel informacij (\texttt{wait()}, \texttt{waitpid()} \dots) o končanem procesu.
	Jedro vedno obdrži informacije končanega procesa, saj jih bo starš morda potreboval.
	\item[\texttt{TASK\_DEAD}] Proces je v postopku brisanja, saj je starš ravno prevzel informacije ``zombi'' procesa.
	Sprememba statusa iz \texttt{TASK\_ZOMBIE} na \texttt{TASK\_DEAD} je pomembna za preprečevanje tveganih stanj (\textit{angl. race condition}), kjer bi lahko druge niti izvedle \texttt{wait()} ali podoben klic na istem procesu.
\end{description}

Poleg spremenljivke \texttt{state} ima proces tudi spremenljivko \texttt{exit\_state}, ki se nastavi ob končanju procesa.
Spremenljivka \texttt{exit\_state} ima dve veljavni stanji: \texttt{EXIT\_ZOMBIE} in \texttt{EXIT\_DEAD}, katerih pomen je enak stanjema \texttt{TASK\_ZOMBIE} in \texttt{TASK\_DEAD} \cite{Bovet_Cesati_2005}.

\begin{figure}[h!]
	\begin{center}
		\begin{tikzpicture}[align=center, font=\small]
			\node[ellipse, draw=black, very thick] (created) {Ustvarjen\\z \texttt{fork()}};
			\node[ellipse, draw=black, very thick, below=of created] (ready) {Pripravljen\\\texttt{TASK\_RUNNING}};
			\node[right=of ready] (dummy) {};
			\node[ellipse, draw=black, very thick, right=of dummy] (running) {V izvajanju\\\texttt{TASK\_RUNNING}};
			\node[ellipse, draw=black, very thick, below=2.5cm of dummy] (waiting) {Čakajoč\\\texttt{TASK\_INTERRUPTIBLE} ali\\\texttt{TASK\_UNINTERRUPTIBLE}};
			\node[ellipse, draw=black, very thick, above=of running] (terminated) {``Zombi''\\\texttt{TASK\_ZOMBIE}};
			\node[ellipse, draw=black, very thick, right=of terminated] (dead) {Zaključen\\\texttt{TASK\_DEAD}};
			\draw[line width=0.4mm, -{Stealth[length=3mm]}] (created) -- (ready);
			\draw[line width=0.4mm, -{Stealth[length=3mm]}] (ready) to [out=15, in=165] node [text width=3.5cm,midway,above] {časovnik izbere proces za izvajanje} (running);
			\draw[line width=0.4mm, -{Stealth[length=3mm]}] (running) to [out=195, in=345] node [text width=3.5cm,midway,below] {prekinjen zaradi prioritetnega procesa} (ready);
			\draw[line width=0.4mm, -{Stealth[length=3mm]}] (running) to [in=45, out=270] (waiting);
			\draw[line width=0.4mm, -{Stealth[length=3mm]}] (waiting) to [in=270, out=135] (ready);
			\draw[line width=0.4mm, -{Stealth[length=3mm]}] (running) -- node [midway,right] {\texttt{exit()}} (terminated);
			\draw[line width=0.4mm, -{Stealth[length=3mm]}] (terminated) -- node [midway,above=0.5cm] {\texttt{wait()}} (dead);
		\end{tikzpicture}
	\end{center}
	\caption{Življenjski cikel Linux procesa}
	\label{fig:linux_process_lifecycle}
\end{figure}

Na diagramu \ref{fig:linux_process_lifecycle} vidimo tok stanj v življenjskem ciklu procesa.
Ko je proces ustvarjen, običajno s sistemskim klicem \texttt{fork()}, je ustvarjen v stanju \texttt{TASK\_RUNNING} in čaka, da ga časovnik izbere za izvajanje.
V primeru, da ima časovnik v čakalni vrsti proces z višjo prioriteto, bo zamenjal izvajalni kontekst na prioritetni proces in trenutni proces spet čaka v pripravljenosti.

Ko proces v izvajanju želi počakati na specifični dogodek v sistemu (npr. signal, sprostitev sistemskega vira \dots), preide v čakanje v enem izmed stanj \texttt{TASK\_INTERRUPTIBLE} ali \texttt{TASK\_UNINTERRUPTIBLE}.
Ob prejetju želenega signala proces spet preide v pripravljenost in čaka na izvajanje.

Ko se proces zaključi, običajno s sistemskim klicem \texttt{exit()} ali ob prejetju signala, postane ``zombi'' s prehodom v stanje \texttt{TASK\_ZOMBIE} in čaka da starš prevzame informacije o procesu preden končno preide v stanje \texttt{TASK\_DEAD}

\section{Proces v Windows}

Na najvišjem nivoju abstrakcije, prikazano na sliki \ref{fig:windows_process}, ima vsak proces v Windows \cite{Yosifovich_Russinovich_Solomon_Ionescu_2017}:
\begin{description}
	\item[seznam odprih ročic (\textit{angl. handle})] Te predstavljajo različne sistemske vire kot so semaforji in drugi sinhronizacijski objekti ter datoteke, ki so na voljo vsem nitim procesa.
	\item[varnostni kontekst (\textit{angl. security context})] To je dostopni žeton oz. poverilnica, ki identificira uporabnika, varnostne skupine, privilegije, atribute, pravice, sposobnosti, virtualizacijsko stanje in sejo povezano s procesom.
	\item[ID procesa] Unikatni identifikator procesa.
	\item[vsaj eno nit] Medtem ko je ``prazen'' proces mogoč, to večinoma ni uporabno. 
\end{description}
Implementacijo teh si bomo pobližje ogledali v poglavju \ref{ssec:windows_process:eprocess}.

\begin{figure}[h!]
	\begin{center}
		\includegraphics[width=0.9\textwidth]{images/windows_process.png}
	\end{center}
	\caption{Proces v Windows \cite{Yosifovich_Russinovich_Solomon_Ionescu_2017}}
	\label{fig:windows_process}
\end{figure}

% TODO: Move down
Kot smo videli pri Linux ima tudi Windows proces kazalnik na starša.
Če starš ne obstaja več, se ta informacija ne posodobi, zato je mogoče da se ta informacija nanaša na neobstoječ proces.
Za razliko od Linux pa to ni problem, saj se nič ne zanaša na to informacijo.

Podobno kot pri Linuxu tudi Windows implementira koncept niti, vendar so le-te ločena entiteta od procesa.
Zato je nit nosilka ključnih komponent za izvajanje procesa, kot so:
\begin{itemize}
	\item vsebina nabora CPE registrov, ki predstavljajo stanje procesorja,
	\item sklada za uporabniški in jedrni izvajalni kontekst,
	\item nitno shrambo (\textit{angl. thread-local storage -- TLS}) in
	\item unikatni identifikator niti (\textit{angl. thread ID}).
\end{itemize}
Poleg tega imajo lahko posamezne niti svoj ločen varnostni kontekst.
Implementacijo niti si bomo bolj podrobno ogledali v poglavju \ref{ssec:windows_process:ethread}.

Windows s konceptom vlaken (\textit{angl. fiber}) ponuja še globljo delitev izvajalnega konteksta.
Niti lahko ljubkovalno imenujemo tudi lahkotne niti, saj se obnašajo podobno, vendar so implementirane v celoti v uporabniškem kontekstu in se jih jedro sploh ne zaveda \cite{Yosifovich_Russinovich_Solomon_Ionescu_2017}.
Uporabljajo se takrat, ko želi aplikacija implementirati lastni časovnih za razvrščanje izvajanja.
Podobno funkcionalnost lahko vidimo pri starejših implementacijah niti v Linux sistemih.
Ker vlakna niso implementirana v jedru, za naše namene niso zanimiva, vendar je vredno izpostaviti prvorazredno podporo za to funkcionalnost.

Poleg vlaken Windows ponuja tudi uporabniško razvrščevalne niti (\textit{angl. user-mode scheduling thread -- UMS thread}).
Njihova funkcionalnost je podobna vlakom.
Za razliko od vlaken pa se jedro zaveda UMS niti in razlikuje med njimi ko pride do sistemskih klicev, saj vsaka prejme lasten jedrni kontekst.

Če pa se pomaknemo nivo višje od procesa, Windows implementira koncept poslov (\textit{angl. job}), ki omogočajo upravljanje in manipulacijo skupine procesov kot enoto.
Posli nekoliko kompenzirajo za pomanjkanje strukturiranega procesnega drevesa in so lahko v mnogi pogledi celo bolj zmogljive kot procesna drevesa UNIX podobnih sistemov \cite{Yosifovich_Russinovich_Solomon_Ionescu_2017}.
% TODO: Dopolni Ch.3-Jobs

\subsection{EPROCESS in KPROCESS} \label{ssec:windows_process:eprocess}

Podobno kot Linux procesni deskriptor ima tudi Windows jedro podatkovne strukture, ki predstavljajo izvajalni proces.
Windows v ta namen uporablja dve strukturi -- \texttt{EPROCESS} in \texttt{KPROCESS} \cite{Yosifovich_Russinovich_Solomon_Ionescu_2017}.
\texttt{EPROCESS} je višje nivojska in se uporablja v izvajalnem kontekstu, torej za spremljanje izvajanja procesa.
\texttt{KPROCESS} pa je jedrna struktura, ki se uporablja v časovniku ter prekinitveni in računovodski kodi jedra.
Strukturi sta povezani preko bloka za nadzor procesa (\textit{angl. Process Control Block -- PCB}) v \texttt{EPROCESS}, ki je kazalnik na strukturo tipa \texttt{KPROCESS} \cite{Yosifovich_Russinovich_Solomon_Ionescu_2017}.

\texttt{EPROCESS}, prikazana na sliki \ref{fig:windows_ethread}, in nanjo vezane podatkovne strukture se hranijo v jedrnem naslovnem prostoru.
Ena izjema temu pravilu je blok izvajalnega okolja procesa (\textit{angl. Process Environment Block -- PEB}), ki se hrani v uporabniškem naslovnem prostoru, saj vsebuje informacije potrebne za izvajanje uporabniške kode \cite{Yosifovich_Russinovich_Solomon_Ionescu_2017}.
Nivo višje upravljalnik izvršilnih objektov enkapsulira vsako instanco strukture \texttt{EPROCESS} v procesni objekt (\texttt{WinObj}).

\begin{figure}[h!]
	\begin{center}
		\includegraphics[width=0.9\textwidth]{images/windows_eprocess.png}
	\end{center}
	\caption{EPROCESS v Windows \cite{Yosifovich_Russinovich_Solomon_Ionescu_2017}}
	\label{fig:windows_eprocess}
\end{figure}

Enako se \texttt{KPROCESS}, prikazana na sliki \ref{fig:windows_kprocess}, in nanjo vezane strukture hranijo v jedrnem naslovnem prostoru.
Za razliko od \texttt{EPROCESS}, je \texttt{KPROCESS} v celoti rezervirana samo za jedro sistema.

\begin{figure}[h!]
	\begin{center}
		\includegraphics[width=0.9\textwidth]{images/windows_kprocess.png}
	\end{center}
	\caption{KPROCESS v Windows \cite{Yosifovich_Russinovich_Solomon_Ionescu_2017}}
	\label{fig:windows_kprocess}
\end{figure}

Poleg \texttt{EPROCESS} in \texttt{KPROCESS} različni podsistemi in njihovi procesi vzdržujejo ločene strukture za proces, ki so v večini primerov ustvarjene po potrebi, običajno ko program naloži specifično sistemsko knjižnico \cite{Yosifovich_Russinovich_Solomon_Ionescu_2017}.

Iz slik \ref{fig:windows_eprocess} in \ref{fig:windows_kprocess} vidimo, da Windows o procesu beleži podobne informacije kot Linux:
\begin{itemize}
	\item unikatni identifikator procesa in starša,
	\item tabelo odprtih ročic,
	\item varnostni kontekst oz. žeton,
	\item procesne števce (računovodske informacije),
	\item reference na niti,
	\item sejo v kateri proces nastopa \dots
\end{itemize}

% Poleg tega proces podsistema Windows (\textit{Client/Server Runtime Subsystem} -- \texttt{csrss.exe}) vzdržuje vzporedno strukturo \texttt{CSR_PROCESS} za vsak proces, ki izvaja Windows program \cite{Yosifovich_Russinovich_Solomon_Ionescu_2017}.
% Prav tako jedrni del podsistema Windows vzdržuje podatkovno strukturo \texttt{W32PROCESS}, ki se ustvari prvič ko nit pokliče katero koli jedrno Windows USER ali GDI funkcijo.
% To se zgodi takoj ko naložimo \texttt{User32.dll} knjižnico, ki jo potrebujejo funkcije kot \texttt{CreateWindow} in \texttt{GetMessage}.

\subsection{ETHREAD in KTHREAD} \label{ssec:windows_process:ethread}

Kot že omenjeno Windows, za razliko od Linux, sledi nitim z ločenimi podatkovnimi strukturami.
Tako kot pri procesih se uporabljata dve strukturi -- \texttt{ETHREAD} in \texttt{KTHREAD} \cite{Yosifovich_Russinovich_Solomon_Ionescu_2017}
Njuni vlogi sta simetrični strukturama \texttt{EPROCESS} in \texttt{KPROCESS}, kjer se \texttt{ETHREAD} uporablja v izvajalnem kontekstu in \texttt{KTHREAD} v jedrnem kontekstu za uvrščanje in prekinitve.
Spet sta strukturi povezani preko bloka za nadzor niti (\textit{angl. Thread Control Block -- TCB}) v \texttt{ETHREAD}, ki je kazalnik na strukturo tipa \texttt{KTHREAD} \cite{Yosifovich_Russinovich_Solomon_Ionescu_2017}.

\texttt{ETHREAD}, prikazana na sliki \ref{fig:windows_ethread}, in nanjo vezane podatkovne strukture se hranijo v jedrnem naslovnem prostoru.
Spet pa je temu pravilu ena izjema -- blok izvajalnega okolja niti (\textit{angl. Thread Environment Block -- TEB}), ki se hrani v uporabniškem naslovnem prostoru, podobno kot PEB za \texttt{EPROCESS}.

\begin{figure}[h!]
	\begin{center}
		\includegraphics[width=0.9\textwidth]{images/windows_ethread.png}
	\end{center}
	\caption{ETHREAD v Windows \cite{Yosifovich_Russinovich_Solomon_Ionescu_2017}}
	\label{fig:windows_ethread}
\end{figure}

Na sliki \ref{fig:windows_ethread_tib} si pobližje pogledamo vsebino TEB in vidimo, da med drugim vsebuje najpomembnejše informacije za izvajanje programa kot so kazalnik na sklad, mejo sklada, lokalno shrambo niti in seveda informacije o starševskem procesu.

\begin{figure}[h!]
	\begin{center}
		\includegraphics[width=0.7\textwidth]{images/windows_ethread_teb.png}
	\end{center}
	\caption{ETHREAD TEB \cite{Yosifovich_Russinovich_Solomon_Ionescu_2017}}
	\label{fig:windows_ethread_tib}
\end{figure}

Enako se \texttt{KTHREAD}, prikazana na sliki \ref{fig:windows_kthread}, in nanjo vezane strukture hranijo v jedrnem naslovnem prostoru.
Za razliko od \texttt{ETHREAD}, je \texttt{KTHREAD} v celoti rezervirana samo za jedro sistema.

\begin{figure}[h!]
	\begin{center}
		\includegraphics[width=0.9\textwidth]{images/windows_kthread.png}
	\end{center}
	\caption{KTHREAD v Windows \cite{Yosifovich_Russinovich_Solomon_Ionescu_2017}}
	\label{fig:windows_kthread}
\end{figure}

Poleg \texttt{ETHREAD} in \texttt{KTHREAD} različni podsistemi in njihovi procesi vzdržujejo ločene strukture za nit \cite{Yosifovich_Russinovich_Solomon_Ionescu_2017}.

Iz slik \ref{fig:windows_ethread}, \ref{fig:windows_ethread_tib} in \ref{fig:windows_kthread} vidimo, da jedro Windows vzdržuje podobne informacije o nitih kot o procesih:
\begin{itemize}
	\item unikatni identifikator niti,
	\item unikatni identifikator in kazalnik na proces kateremu pripada,
	\item varnostni kontekst oz. žeton,
	\item informacije ključne za izvajanje \dots
\end{itemize}
Manjkajo pa reference na deljene vire kot tabela odprtih ročic, seja \dots

\chapter{Upravljanje procesov}

Sistemski klici za upravljanje procesov skrbijo za:
\begin{itemize}
	\item ustvarjanje in končanje procesov,
	\item nalaganje in izvajanje programov,
	\item pridobivanje in nastavljanje atributov procesov,
	\item sinhronizacijo procesov,
	\item dodeljevanje in sproščanje pomnilnika \cite{Silberschatz_Galvin_Gagne_2018}.
\end{itemize}

\section{Sistemski klici v Linux}

\subsection{Ustvarjanje in končanje procesa}

\subsubsection{\texttt{fork}}

Funkcija \texttt{fork()} ustvari nov proces, ki je duplikat izvajajočega procesa.
Nov proces postane otrok klicajočega procesa.

Otroški proces je identična kopija starševskega, z izjemo nekaj ključnih točk:
\begin{itemize}
	\item Identifikator procesa (PID) je unikaten za vsak proces, otrok torej dobi nov PID.
	\item Identifikator starševskega procesa (PPID) postane enak kot PID starševskega (klicajočega) procesa.
	\item Otrok ne deduje starševih pomnilniških ključavnic.
	\item Ponastavijo se števci uporabe virov in CPU časa.
	\item Otrok ne deduje čakajočih signalov.
	\item Otrok ne deduje datotečnih ključavnic asociiranih s procesov. Deduje pa ključavnice asociirane z odprtimi datotekami.
	\item Otrok ne deduje obstoječih asinhronih V/I operacij ali asinhronih V/I kontekstov.
\end{itemize}

\begin{lstlisting}[style=func]
 pid_t fork(void);
\end{lstlisting}

Funkcija ne sprejme nobenih vhodnih parametrov, vrne pa \texttt{0} v otroškem procesu in PID otroškega procesa v starševskem procesu.
V primeru napake funkcija vrne \texttt{-1} starševskemu procesu in ne ustvari otroškega procesa.

Vredno je pripomniti, da je \texttt{fork} v Linux implementiran preko tako imenovanega ``copy-on-write'' mehanizma.
To pomeni, da kreiranje novega procesa zahteva minimalno kopiranje v pomnilniku, saj se to izvede šele ko je re potrebno -- tj. ko proces začne spreminjati pomnilnik.

\subsubsection{\texttt{execve}}

Ko želimo zagnati nov proces, ki izvaja drug program, takoj za klicem \texttt{fork} kličemo funkcijo \texttt{execve()}.
\texttt{execve()} zamenja programski spomin trenutnega procesa z programom v podani datoteki.
Prav tako na novo inicializira sklad, kopico in podatkovne segmente.

Če ima program (datoteka), ki ga nalagamo, nastavljen ``setuid'' bit, se efektivni uporabniški ID procesa spremeni na ID lastnika datoteke.
Enako se zgodi če ima program nastavljen ``getgid'' bit, samo da se ta navezuje na ID skupine.
Te pravili pa imata nekaj izjem, ki se navezujejo na atribute klicajoče niti.

\texttt{execve} ohrani vse atribute procesa z nekaj izjemami:
\begin{itemize}
	\item Dispozicije (akcije) signalov se ponastavijo.
	\item Preslikani pomnilniki se ne ohranijo.
	\item Regije deljenega pomnilnika, vrste, semaforji, časovniki in ključavnice se ne ohranijo.
	\item Vse niti razen kicajoče so ustavljene.
\end{itemize}

\begin{lstlisting}[style=func]
 int execve(const char *pathname, char *const _Nullable argv[], char *const _Nullable envp[]);
\end{lstlisting}

Funkcija sprejme tri argumente: pot do programa, argumente programa (opcijski) in okoljske spremenljivke programa (opcijski).
Ob uspešni izvedbi funkcija ne vrne ničesar, saj se začne izvajati novo naloženi program, v primeru napake pa vrne \texttt{-1}.

\subsubsection{\texttt{exit}}

Ko želimo trenutni proces ustaviti, pokličemo funkcijo \texttt{exit()}, ki zaključi in uniči proces.
Sistem zapre odprte dokumente, osirotene (otroške) procese podeduje začetni proces (PID 1) in staršu pošlje signal SIGCHLD.

\begin{lstlisting}[style=func]
 void exit(int status);
\end{lstlisting}

Funkcija sprejme argument \textit{status}, katerega spodnji oktet oz. bajt je posredovan staršu.
Funkcija v nobenem primeru ne vrne ničesar, saj se proces zaključi.

Več o signalu \texttt{SIGCHLD} in posredovanju statusa staršu v poglavju \ref{ssec:linux_syscalls:waiting}.

\subsubsection{\texttt{kill}}

Kadar pa želimo ustaviti drug proces, uporabimo funkcijo \texttt{kill}, ki želenemu procesu pošlje signal.

Signali so neke vrste programske prekinitve, ki se uporabljajo za obveščanje procesa o asinhronih dogodkih. % https://faculty.cs.niu.edu/~hutchins/csci480/signals.htm
Linux implementira 31 različnih signalov, vsakega z eno izmed petih privzetih dispozicij oz. akcij, ki se izvedejo ob prejemu signala.
Možne privzete dispozicije so: končanje procesa, ignoriranje signala, končanje procesa in izpis sistemskega stanja (\textit{angl. core dump}), zaustavitev procesa ali nadaljevanje procesa, če je trenutno ustavljen.
Proces lahko, z izjemo signalov \texttt{SIGKILL} in \texttt{SIGSTOP}, spremeni privzeto dispozicijo signala in implementira lastno prekinitveno funkcijo.

Za naše potrebe je pomembnih le naslednjih nekaj signalov:
\begin{itemize}
	\item \texttt{SIGCHLD}
	      \begin{description}
	      	\item Opozori proces o zaustavitvi in nadaljevanju izvajanja ali končanju otroškega procesa.
	      	\item Proces privzeto ignorira signal.
	      \end{description}
	\item \texttt{SIGTERM}
	      \begin{description}
	      	\item Sporoči procesu naj se mirno konča.
	      	\item Isto sporoča sorodni signal \texttt{SIGINT}, vendar je ta namenjen za uporabniške zahteve in ga običajno lahko proži uporabnik iz interaktivne seje.
	      	\item Enako velja za \texttt{SIGHUP}, vendar le-tega proži sistem ob zaključku uporabniške seje.
	      \end{description}
	\item \texttt{SIGKILL}
	      \begin{description}
	      	\item Prisilno konča proces. Kot že omenjeno, proces ne more spremeniti privzete dispozicije.
	      \end{description}
	\item \texttt{SIGSTOP}
	      \begin{description}
	      	\item Sporoči procesu naj se zaustavi. Kot že omenjeno, proces ne more spremeniti privzete dispozicije.
	      	\item Sorodni signal \texttt{SIGTSTP} pa je namenjen specifično uporabniški zahtevi za zaustavitev. Glavna razlika je, da program lahko zanj implementira lastno prekinitveno funkcijo.
	      \end{description}
	\item \texttt{SIGCONT}
	      \begin{description}
	      	\item Sporoči procesu naj nadaljuje izvajanje, če je zaustavljen.
	      \end{description}
	\item \texttt{SIGUSR1} in \texttt{SIGUSR2}
	      \begin{description}
	      	\item Posebna signal, ki jih lahko program uporabi za lastne potrebe sporočanja, saj jih sistem ne proži.
	      	\item Privzeto signala sprožita končanje procesa.
	      \end{description}
\end{itemize}

\begin{lstlisting}[style=func]
 int kill(pid_t pid, int sig);
\end{lstlisting}

Funkcija sprejme dva argumenta: identifikator ciljnega procesa in signal, ki ga želimo poslati.
Ob uspešni izvedbi vrne \texttt{0}, v primeru napake pa \texttt{-1}.

\subsection{Čakanje} \label{ssec:linux_syscalls:waiting}

Kadar želimo sinhronizirati starševski proces z otroškim procesom, blokiramo izvajanje starševskega procesa, dokler ciljni proces ne pošlje signala o spremembi stanja.
Specifično, čakamo signal \texttt{SIGCHLD}, ki ga otroški proces pošlje ob končanju ter zaustavitvi ali nadaljevanju izvajanja procesa.

Ko se otroški proces zaključi, postane ``zombi'' proces.
Jedro o ``zombi'' procesih vzdržuje minimalno zbirko informacij kot so PID, status in informacije o uporabi sistemskih virov, da lahko starševski proces kasneje počaka na proces in pridobi omenjene informacije.
Prav tako to pomeni, da ``zombi'' procesi zasedajo mesto v jedrni tabeli procesov in lahko potencialno onemogočijo kreiranje novih procesov.
V primeru, da se starševski proces konča, njegove ``zombi'' procese podeduje proces \texttt{init}, ki jih počaka in s tem sprosti zasedene vire.

\subsubsection{\texttt{wait}, \texttt{waitpid}}

Najbolj preprosta funkcija za čakanje, \texttt{wait}, blokira trenutno nit dokler se eden izmed otroških procesov ne konča.
Funkcijsko je ekvivalenten klicu \verb|waitpid(-1, &wstatus, 0);|.

\begin{lstlisting}[style=func]
 pid_t wait(int *_Nullable wstatus);
\end{lstlisting}

Funkcija sprejme kazalnik na spremenljivko tipa integer, kamor zapiše statusne informacije končanega otroškega procesa.
Ob uspešni izvedbi vrne PID končanega otroškega procesa, v primeru napake pa \texttt{-1}.

Status procesa vrnjen preko argumenta \texttt{wstatus}, lahko dekodiramo z uporabo sledečih makrojev:
\begin{itemize}
	\item \texttt{WIFEXITED(wstatus)}
	      \begin{description}
	      	\item vrne \texttt{true}, če je otrok zaključil običajno (klic \texttt{exit} ali \texttt{return} iz funkcije \texttt{main()})
	      \end{description}
	\item \texttt{WEXITSTATUS(wstatus)}
	      \begin{description}
	      	\item vrne izhodni status otroka
	      \end{description}
	\item \texttt{WIFSIGNALED(wstatus)}
	      \begin{description}
	      	\item vrne \texttt{true}, če se je proces zaključil zaradi signala
	      \end{description}
	\item \texttt{WTERMSIG(wstatus)}
	      \begin{description}
	      	\item vrne številko signala, ki je povzročil zaključek procesa
	      \end{description}
	\item \texttt{WCOREDUMP(wstatus)}
	      \begin{description}
	      	\item vrne \texttt{true}, če je proces ustvaril izpis sistemskega stanja
	      \end{description}
	\item \texttt{WIFSTOPPED(wstatus)}
	      \begin{description}
	      	\item vrne \texttt{true}, če je bil otroški proces ustavljen s signalom (mogoče samo pri klicu \texttt{waitpid} z opcijo \texttt{WUNTRACED})
	      \end{description}
	\item \texttt{WSTOPSIG(wstatus)}
	      \begin{description}
	      	\item vrne številko signala, ki je povzročil ustavitev izvajanja procesa
	      \end{description}
	\item \texttt{WIFCONTINUED(wstatus)}
	      \begin{description}
	      	\item vrne \texttt{true}, če proces nadaljuje izvajanje zaradi prejetega signala \texttt{SIGCONT}
	      \end{description}
\end{itemize}

Bolj specifična funkcija, \texttt{waitpid}, blokira trenutno nit dokler otrok s podanim PID ne spremeni stanja.

\begin{lstlisting}[style=func]
	pid_t waitpid(pid_t pid, int *_Nullable wstatus, int options);
\end{lstlisting}

Funkcija sprejme tri atribute, od katerih se \texttt{wstatus} obnaša enako kot pri \texttt{wait}.

Argument \texttt{pid} določa ciljni proces in sprejme štiri različne tipe vhoda:

\begin{tabular}{ p{0.1\linewidth} p{0.82\linewidth} }
	$< -1$ & določa kateri koli otroški proces, katerega skupinski ID je enak absolutni vrednosti argumenta \\
	$-1$   & določa kateri koli otroški proces (kot \texttt{wait})                                          \\
	$0$    & določa kateri koli otroški proces katerega skupinski ID je enak trenutnemu procesu             \\    
	$> 0$  & določa točen PID procesa                                                                       
\end{tabular}

Privzeto funkcija čaka samo zaključene otroške procese, vendar to lahko spremenimo s podajanjem \texttt{options} parametra, za katerega imamo tri opcije, ki jih lahko združujemo z logično OR funkcijo:

\begin{tabular}{ p{0.18\linewidth} p{0.72\linewidth} }
	\texttt{WNOHANG}    & ne blokira niti, če se ciljni otroški proces ni zaključil                                     \\
	\texttt{WUNTRACED}  & vrne rezultat tudi, če se je izvajanje procesa ustavi                                           \\
	\texttt{WCONTINUED} & vrne rezultat tudi, če ustavljen proces nadaljuje izvajanje ob prejemu signala \texttt{SIGCONT} 
\end{tabular}

\subsubsection{\texttt{waitid}}

Kadar potrebujemo še bolj specifičen nadzor nad čakano spremembo statusa, uporabimo funkcijo \texttt{waitid}.

\begin{lstlisting}[style=func]
	int waitid(idtype_t idtype, id_t id, siginfo_t *infop, int options);
\end{lstlisting}

Funkcija sprejme štiri argumente.
Argument \texttt{idtype} in \texttt{id} določata ciljne otroške procese:
\begin{itemize}
	\item \texttt{idtype == P\_PID}
	      \begin{description}
	      	\item čaka proces katerega PID je enak argumentu \texttt{id}
	      \end{description}
	\item \texttt{idtype == P\_PIDFD}
	      \begin{description}
	      	\item čaka proces na katerega se nanaša PID opisnik (\textit{angl. PID file descriptor}) podan v argumentu \texttt{id}
	      \end{description}
	\item \texttt{idtype == P\_PGID}
	      \begin{description}
	      	\item čaka kateri koli proces, katerega skupinski ID je enak argumentu \texttt{id}
	      \end{description}
	\item \texttt{idtype == P\_ALL}
	      \begin{description}
	      	\item čaka kateri koli otroški proces in ignorira argument \texttt{id}
	      \end{description}
\end{itemize}

Željeno spremembo statusa določamo z argumentom \texttt{options}, kjer lahko sledeče opcije združujemo z logično OR operacijo:
\begin{itemize}
	\item \texttt{WEXITED}
	      \begin{description}
	      	\item čaka zaključeno stanje
	      \end{description}
	\item \texttt{WSTOPPED}
	      \begin{description}
	      	\item čaka zaustavitev izvajanja procesa zaradi signala
	      \end{description}
	\item \texttt{WCONTINUED}
	      \begin{description}
	      	\item čaka nadaljevanje izvajanja procesa zaradi signala
	      \end{description}
	\item \texttt{WNOHANG}
	      \begin{description}
	      	\item ne blokira niti, če se ciljni otroški proces ni zaključil
	      \end{description}
	\item \texttt{WNOWAIT}
	      \begin{description}
	      	\item ne sprosti stanja procesa, da lahko kasneje spet pridobimo statusne informacije
	      \end{description}
\end{itemize}

\subsection{Niti}

\subsubsection{\texttt{clone}, \texttt{clone3}}

Kreiranje niti je v osnovi enako kreiranju novega procesa.
Glavna razlika je, da si procesa delita dele izvajalnega konteksta.
Funkciji \texttt{clone} in \texttt{clone3} omogočata kreiranje novega procesa z nadzorom nad deljenimi konteksti kot so naslovni prostor, tabela datotečnih opisnikov, tabela prekinitvenih funkcij signalov \dots

\begin{lstlisting}[style=func]
 int clone(int (*fn)(void *_Nullable), void *stack, int flags, void *_Nullable arg, pid_t *_Nullable parent_tid, void *_Nullable tls, pid_t *_Nullable child_tid);
\end{lstlisting}

Funkcija sprejme sedem argumentov, ki določajo kje bo nov proces pričel z izvajanjem in katere kontekste si bo delil s starševskim procesom.
Nov proces bo začel z izvajanje v funkciji podani v argumentu \texttt{*fn}, ki prejme tudi argument \texttt{*arg}.
Ročno mu moramo pripraviti sklad in podati začetni ali končni naslov (\texttt{*stack}), za kar moramo vedeti ali sklad raste navzgor ali navzdol na danem sistemu.
Pri rezervaciji sklada se običajno poslužujemo funkcije \texttt{mmap}, ki nam omogoča nadzor na prostorom, ki ga rezerviramo.

Preostali argumenti nadzirajo deljenje konteksta in starševski proces nove niti.
Predvsem pomemben je argument \texttt{flags}, kjer lahko opcije združujemo z logično OR operacijo.
Ker je seznam opcij zelo obsežen in v nekaterih primerih zahteva nadalnje obrazložitve, se bomo osredotočili samo na najbolj pomembne zastavice za našo rabo:
\begin{itemize}
	\item \texttt{CLONE\_CHILD\_CLEARTID}
	      \begin{description}
	      	\item prepiše otrokov TID (\textit{angl. thread ID}) na lokaciji \texttt{*child\_tid}, ko se proces zaključi, in zbudi futex (\textit{uporabniški mutex}) na naslovu
	      \end{description}
	\item \texttt{CLONE\_CHILD\_SETTID}
	      \begin{description}
	      	\item hrani otrokov TID na lokaciji \texttt{*child\_tid} v otrokovem pomnilniku
	      \end{description}
	\item \texttt{CLONE\_CLEAR\_SIGHAND}
	      \begin{description}
	      	\item prepiše prekinitvene funkcije za signale v otroškem procesu
	      \end{description}
	\item \texttt{CLONE\_FILES}
	      \begin{description}
	      	\item otrok si bo delil tabelo datotečnih opisnikov s staršem
	      \end{description}
	\item \texttt{CLONE\_FS}
	      \begin{description}
	      	\item otrok si bo delil informacije o datotečnem sistemu (korenski imenik, trenutni delovni imenik in umask) s staršem
	      \end{description}
	\item \texttt{CLONE\_IO}
	      \begin{description}
	      	\item otrok si bo delil vhodno-izhodni konteks s staršem
	      \end{description}
	\item \texttt{CLONE\_PARENT}
	      \begin{description}
	      	\item starš novega procesa bo enak staršu trenutnega procesa
	      \end{description}
	\item \texttt{CLONE\_PARENT\_SETTID}
	      \begin{description}
	      	\item hrani otrokov TID na lokaciji \texttt{*parent\_tid} v starševem pomnilniku
	      \end{description}
	\item \texttt{CLONE\_SIGHAND}
	      \begin{description}
	      	\item otrok si bo delil prekinitvene funkcije za signale s staršem
	      \end{description}
	\item \texttt{CLONE\_SYSVSEM}
	      \begin{description}
	      	\item otrok si bo delil seznam nastavitev za System V semaforje
	      \end{description}
	\item \texttt{CLONE\_THREAD}
	      \begin{description}
	      	\item otrok bo del iste skupine niti kot trenutni proces; vse niti si delijo isti PID oz. TGID (\textit{angl. thread group ID})
	      \end{description}
	\item \texttt{CLONE\_VM}
	      \begin{description}
	      	\item otrok si bo delil naslovni prostor s staršem
	      \end{description}
\end{itemize}

Izmed opisanih opcij sta za niti najbolj ključni \texttt{CLONE\_VM}, saj je kopiranje pomnilnika najbol časovno intenziven del ustvarjanja novega procesa, in \texttt{CLONE\_THREAD}, saj pove sistemu, da je nova nit del obstoječega procesa.
Vidimo pa lahko manjšo težavo pri opciji \texttt{CLONE\_THREAD}.
Nova nit si namreč deli isti PID in posledično tudi PPID, kar pomeni da trenutna nit ne prejme signala \texttt{SIGCHLD} ob končanju nove niti in moramo to komunicirati na nek drug način.
Seveda je tu najbolj preprosta rešitev uporaba deljenega pomnilnika, vendar implementacija lahko hitro postane nepregledna.
Zato v praksi uporabljamo POSIX niti, implementirane v glibc knjižnici z uporabo sistemskih klicev \texttt{clone} in \texttt{clone3}.

\subsection{Spanje}

\subsubsection{\texttt{nanosleep}}

Kadar želimo proces zaustaviti za določeno količino časa, nam Linux ponuja funkcijo \texttt{nanosleep}.
Tako lahko ustavimo izvajanje in sprostimo procesor dokler se podani čas ne izteče ali pa je spanje prekinjeno zaradi prejema signala.

\begin{lstlisting}[style=func]
 int nanosleep(const struct timespec *duration, struct timespec *_Nullable rem);
\end{lstlisting}

Funkcija sprejme dva atributa, \texttt{duration}, kjer določimo čas spanja v nanosekundah (veljavne so vrednosti od 0 do 999999999), in \texttt{rem}, kjer nam funkcija sporoči preostali čas spanja v primeru prekinitve.
Pri uspešni izvedbi funkcija vrne \texttt{0}, v primeru prekinitve spanja s signalom ali druge napake pa \texttt{-1}.

\subsection{Pridobivanje sistemskih informacij}

Linux nam za pridobivanje informacij ponuja nekaj osnovnih funkcij, ki nam podajo identifikatorje procesa.

\subsubsection{\texttt{getpid}, \texttt{getppid}}

Najbolj osnovni informaciji o procesu sta ID procesa (PID) in ID starševskega procesa (PPID), ki jih lahko dobimo preko funkcij \texttt{getpid} in \texttt{getppid}.

\begin{lstlisting}[style=func]
 pid_t getpid(void);
 pid_t getppid(void);
\end{lstlisting}

\subsubsection{\texttt{getpgid}, \texttt{setpgid}}

Pri klicih \texttt{waitpid} in \texttt{waitid} smo na hitro omenili koncept skupine procesov, ki jo identificira ID skupine (PGID).
Skupino procesa lahko pridobimo s funkcijo \texttt{getpgid}, ki ji podamo PID ciljnega procesa ali \texttt{0} za trenutni proces.
Pri spreminjanju skupine pa se poslužujemo \texttt{setpgid}, ki ji spet podamo PID ciljnega procesa ter nov PGID.
Pri tem je pogoj, da sta proces in skupina del iste seje.

\begin{lstlisting}[style=func]
 int setpgid(pid_t pid, pid_t pgid);
 pid_t getpgid(pid_t pid);
\end{lstlisting}

\subsubsection{\texttt{gettid}}

Kot smo videli pri \texttt{clone}, imajo niti poleg PID še ID niti (TID), ki jih razlikuje.
TID lahko pridobimo s funkcijo \texttt{gettid}, ki v primeru procesa z eno nitjo preprosto vrne PID.

\begin{lstlisting}[style=func]
 pid_t gettid(void);
\end{lstlisting}

\section{Sistemski klici v Windows}

\subsection{Ustvarjanje in končanje procesa}

\subsection{Čakanje}

\subsection{Niti}

\subsection{Spanje}

\subsection{Pridobivanje sistemskih informacij}

\section{Primerjava sistemov}

\chapter{Zaključek}

%\cleardoublepage
%\addcontentsline{toc}{chapter}{Literatura}

\printbibliography[heading=bibintoc,title={Literatura}]


\end{document}
